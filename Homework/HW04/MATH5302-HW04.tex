% Standard Article Definition
\documentclass[]{article}

% Page Formatting
\usepackage[margin=1in]{geometry}
\setlength\parindent{0pt}

% Graphics
\usepackage{graphicx}

% Math Packages
\usepackage{physics}
\usepackage{amsmath, amsfonts, amssymb, amsthm}
\usepackage{mathtools}

% Extra Packages
\usepackage{listings}
\usepackage{hyperref}

% Section Heading Settings
\usepackage{enumitem}
\renewcommand{\theenumi}{\alph{enumi}}
\renewcommand*{\thesection}{Problem \arabic{section}}
\renewcommand*{\thesubsection}{\alph{subsection})}
\renewcommand*{\thesubsubsection}{}%\quad \quad \roman{subsubsection})}

%Custom Commands
\newcommand{\Rel}{\mathcal{R}}
\newcommand{\R}{\mathbb{R}}
\newcommand{\C}{\mathbb{C}}
\newcommand{\N}{\mathbb{N}}
\newcommand{\Z}{\mathbb{Z}}
\newcommand{\Q}{\mathbb{Q}}

\newcommand{\toI}{\xrightarrow{\textsf{\tiny I}}}
\newcommand{\toS}{\xrightarrow{\textsf{\tiny S}}}
\newcommand{\toB}{\xrightarrow{\textsf{\tiny B}}}

\newcommand{\divisible}{ \ \vdots \ }
\newcommand{\st}{\ : \ }

% Theorem Definition
\newtheorem{definition}{Definition}
\newtheorem{assumption}{Assumption}
\newtheorem{theorem}{Theorem}
\newtheorem{lemma}{Lemma}
\newtheorem{proposition}{Proposition}
\newtheorem{example}{Example}


%opening
\title{MATH 5302 Elementary Analysis II - Homework 2}
\author{Jonas Wagner}
\date{2022, Febuary 9\textsuperscript{th}}

\begin{document}

\maketitle

% Problem 1 ----------------------------------------------
\section{}
Show that\[
    B(\alpha,\beta) = \int_0^1 x^{\alpha-1} (1-x)^{\beta-1} \dd{x}
\] is well-defined for $\alpha > 0$ and $\beta > 0$.

\begin{definition} \label{def:well_defined}
    The improper integral \[
        \int_{0}^{a} f(x) \dd{x}
    \] is \emph{\underline{well-defined}} iff\[
        \lim_{\epsilon \to 0} \int_{0}^{a} f(x) \dd{x}
    \] exists.
\end{definition}

\begin{definition} \label{def:gamma_fun}
    The \underline{\emph{gamma function}} $\Gamma(\alpha)$ is defined as \[
        \Gamma(\alpha) = \int_{0}^{\infty} e^{-x} x^{\alpha - 1} \dd{x}
    \] for $0 < \alpha < \infty$.
\end{definition}

\begin{definition} \label{def:beta_fun}
    The \underline{\emph{beta function}} $B(\alpha, \beta)$ is defined as \[
        B(\alpha,\beta) = \int_0^1 x^{\alpha-1} (1-x)^{\beta-1} \dd{x}
            = \cfrac{\Gamma(\alpha)\Gamma(\beta)}{\Gamma(\alpha + \beta)}
    \] for $\alpha>0$ and $\beta>0$.
\end{definition}


% \begin{theorem}\label{thm:comp_test}
%     \textbf{Comparison Test:}
%     Let $f,g:[a,b) \to \R$ be two functions such that 
%     (i) $f(x)$ and $g(x)$ are integrable on $[a,A]\subset[a,b)$, for $a<A<b$;
%     (ii) There exists $a < M < b$ such that $0\leq f(x) \leq g(x)$ forall $x \in [M,b)$.
%     Then,\begin{enumerate}
%         \item If $\int_a^b g(x) \dd{x}$ converges then $\int_a^b f(x) \dd{x}$ also converges;
%         \item If $\int_a^b f(x) \dd{x}$ diverges then $\int_a^b f(x) \dd{x}$ also diverges.
%     \end{enumerate}
% \end{theorem}

\begin{theorem}\label{thm:limit_comp_test}
    \textbf{Limit Comparison Test:}
    Let $f,g:[a,b) \to \R$ be two functions such that 
    (i) $f(x)$ and $g(x)$ are integrable on $[a,A]\subset[a,b)$, for $a<A<b$;
    (ii) There exists $a \leq K \leq b$ such that $\lim_{x\to b^{-}} \frac{f(x)}{g(x)} = K$.
    Then,\begin{enumerate}
        \item If $0< K < \infty$, then $\int_a^b g(x) \dd{x}$ converges iff $\int_a^b f(x)$ converges.
        \item If $K = 0$, then $\int_a^b g(x)$ converges implies $\int_a^b f(x) \dd{x}$ converges.
        \item If $K \infty 0$, then $\int_a^b g(x)$ divergent implies $\int_a^b f(x) \dd{x}$ divergent.
    \end{enumerate}
\end{theorem}

% \begin{theorem}\label{thm:cauchy_criterion}
%     \textbf{Cauchy Criterion:}
%     Let $f:[a,b) \to \R$ be a function integrable on every $[a,A]\subset[a,b)$, for $a<A<b$.
%     Then the imporper integral $\int_a^b f(x) \dd{x}$ converges if and only if\[
%         \forall_{\eta> 0} \exists_{\epsilon \in (0, b-a)} \st 
%         A, B \in (b-\epsilon,b) 
%         \implies \abs{\int_{A}^{B} f(x) \dd{x}} < \eta
%     \]
% \end{theorem}

\begin{theorem}
    The improper integral that defines the beta function \[
        B(\alpha,\beta) = \int_0^1 x^{\alpha-1} (1-x)^{\beta-1} \dd{x}
    \] is well-defined for $\alpha > 0$ and $\beta > 0$. 
    \begin{proof}
        The integrand of $B(\alpha,\beta)$, \[
            b(\alpha,\beta) = x^{\alpha-1} (1-x)^{\beta - 1}
        \] is not strictly bounded $\forall_{\alpha,\beta>0}$, but this is not necessary for convergence. 
        $\forall{\alpha,\beta \in [0,\infty)}$ the $b(\alpha,\beta)$ is bounded. 
        This makes $B(\alpha,\beta)$ a proper integral which is therefore convergent.

        In the other case the integrand is not bounded, but the improper integral still converges. 
        $\forall_{\alpha \in (0,1)}$ then $b(\alpha,\beta)$ is unbounded at $x = 0$.
        Similarly, $\forall_{\beta \in (0,1)}$ then $b(\alpha,\beta)$ is unbounded at $x = 1$.        

        The beta function can instead be split up into two parts:\[
            B(\alpha,\beta) 
                = \int_{0}^{c} x^{\alpha-1} (1-x)^{\beta-1} \dd{x}
                + \int_{c}^{1} x^{\alpha-1} (1-x)^{\beta-1} \dd{x}
        \] where $c \in (0,1)$.

        For the first improper integral, $\int_{0}^{c} x^{\alpha-1} (1-x)^{\beta-1} \dd{x}$
        a discontinuity exists at $x = 0$ for $\alpha \in (0,1)$. 
        Using the Limit Comparison Test from Theorem \ref{thm:limit_comp_test} with $g(x) = x^{\alpha - 1}$, 
        \begin{align*}
            \lim_{x\to 0^{+}} \frac{f(x)}{g(x)} 
                &= \lim_{x\to 0^{+}} \frac{x^{\alpha-1} (1-x)^{\beta-1}}{x^{\alpha - 1}}\\
                &= \lim_{x\to 0^{+}} (1-x)^{\beta-1}\\
                &= 1 \neq 0
        \end{align*}
        Which then implies that $\int_{0}^{c} x^{\alpha-1} (1-x)^{\beta-1} \dd{x}$ 
        converges $\forall_{\alpha,\beta > 0}$.

        For the second improper integral, $\int_{c}^{1} x^{\alpha-1} (1-x)^{\beta-1} \dd{x}$
        a discontinuity exists at $x = 1$ for $\beta \in (0,1)$. 
        Using the Limit Comparison Test from Theorem \ref{thm:limit_comp_test} with $g(x) = (1-x)^{\beta - 1}$, 
        \begin{align*}
            \lim_{x\to 1^{-}} \frac{f(x)}{g(x)} 
                &= \lim_{x\to 1^{-}} \frac{x^{\alpha-1} (1-x)^{\beta-1}}{(1-x)^{\beta - 1}}\\
                &= \lim_{x\to 1^{-}} x^{\alpha - 1}\\
                &= 1 \neq 0
        \end{align*}
        Which then implies that $\int_{c}^{1} x^{\alpha-1} (1-x)^{\beta-1} \dd{x}$ 
        converges $\forall_{\alpha,\beta > 0}$.

        Together, the convergence of $\int_{0}^{c} x^{\alpha-1} (1-x)^{\beta-1} \dd{x}$ 
        and $\int_{c}^{1} x^{\alpha-1} (1-x)^{\beta-1} \dd{x}$ 
        implies that $B(\alpha,\beta)$ converges $\forall_{\alpha,\beta>0}$ and therefore $B(\alpha,\beta)$ is well defined.
    \end{proof}
\end{theorem}

% Problem 2 -------------------------------------------
\newpage
\section{}
Show that $f$ if Riemann integrable on $[a,b]$, then \[
    \lim_{\epsilon\to 0^{+}} \int_{a}^{b - \epsilon} f(x) \dd{x} = \int_{a}^{b} f(x) \dd{x}
\]

\begin{definition}\label{def:Riemann_int}
    Let $f: [a,b] \to \R$ be bounded on $[a,b]$.
    \begin{enumerate}
        \item A \emph{\underline{Partition}} of $[a,b]$ is any ordered $P \subset [a,b]$ given as \[
            P = \qty{a = x_0 < x_1 < \cdots < x_n < b}
        \]
        \item A \emph{\underline{Mesh}} of partition $P$, $\text{mesh}(P)$, is the maximum length of the subintervals in $P$. 
        (i.e) For $P  = \qty{a = x_0 < x_1 < \cdots < x_n < b}$,\[
            \text{mesh}(P) = \max[\qty{x_{i} - x_{i-1} \st i = 1, 2, \dots, n}]
        \]
        \item A \emph{\underline{Riemann Sum}} of $f$ associated with partition $P$, $S(f,P)$, is the sum defined as \[
            \sum_{i=1}^{n} f(x_i^{*}) (x_i - x_{i-1})
        \] where the specific $x_{i}^{*} \in [x_{i-1}, x_{i}]$ is arbitrary.
        \item $f$ is considered \emph{\underline{Riemann Integrable}} on $[a,b]$ if\[
            \exists_{r} \forall_{\epsilon>0} \exists_{\delta>0} 
            \st \forall_{S(f,P) \st \text{mesh}(P) < \delta} \implies \abs{S(f,P) - r} < \epsilon
        \] where the number $r$ is considered the \emph{\underline{Riemann Integral}} of $f$ on $[a,b]$, $\mathcal{R}\int_{a}^{b} f$.
    \end{enumerate}
\end{definition}

\begin{theorem}
    Let $f : [a,b] \to \R$.
    $f$ being Riemann integrable on $[a,b]$ implies that \[
        \lim_{\epsilon\to 0^{+}} \int_{a}^{b - \epsilon} f(x) \dd{x} = \int_{a}^{b} f(x) \dd{x}
    \]
    \begin{proof}
        By the definition of a function being Riemann Integrable, Definition \ref{def:Riemann_int}, it is known that $f$ must be bounded.
        From this fact, the limit described will always exists and an asymptote at the boundary would not be a concern.
        Using the construction of Riemann sum itself, 
        \begin{align*}
            \lim_{\epsilon\to 0^{+}} \int_{a}^{b - \epsilon} f(x) \dd{x} 
                &= \lim_{\epsilon \to 0^{+}} r \st \forall_{P = \qty{a = x_0 < x_1 < \cdots < x_n < b - \epsilon}}
                \forall_{\epsilon_0>0} \exists_{\delta>0} 
                \st \forall_{S(f,P) \st \text{mesh}(P) < \delta} \implies \abs{S(f,P) - r} < \epsilon_0\\
                &= r \st \forall_{P = \qty{a = x_0 < x_1 < \cdots < x_n < b}} \forall_{\epsilon_0>0} \exists_{\delta>0} 
                \st \forall_{S(f,P) \st \text{mesh}(P) < \delta} \implies \abs{S(f,P) - r} < \epsilon_0\\
            \Aboxed{
                \lim_{\epsilon\to 0^{+}} \int_{a}^{b - \epsilon} f(x) \dd{x} 
                    &= \mathcal{R} \int_a^b f = \int_{a}^{b} f(x) \dd{x}
            }
        \end{align*}
    \end{proof}
\end{theorem}

% Problem 3 -------------------------------------------
\newpage
\section{}
Evaluate $\int_{0}^{1} (1-x^{\frac{2}{3}})^{\frac{3}{2}} \dd{x}$.
Hint: Express the integral in terms of the gamma function first.

\begin{example}
    Let \[
        F(x) = \int_{0}^{1} (1-x^{\frac{2}{3}})^{\frac{3}{2}} \dd{x}
    \] This can be simplified using u-substitution.
    Let \[
        u = x^{\frac{2}{3}}
    \] then \[
        \dd{u} = \frac{2}{3} x^{-\frac{1}{3}} \dd{x}
    \] and \[
        \dd{x} = \frac{3}{2} x^{\frac{1}{3}} \dd{u}
    \]
    The bounds are found as \[
        0 = u(a) = a^{\frac{2}{3}} \implies a = 0^{\frac{3}{2}} = 0
    \] and \[
        1 = u(b) = b^{\frac{2}{3}} \implies b = 1^{\frac{3}{2}} = 1
    \]
    
    \begin{align*}
        F(x) 
            &= \int_{0}^{1} \qty(1 - x^{\frac{2}{3}} )^{\frac{3}{2}} 
                        \qty(\frac{3}{2} x^{\frac{1}{3}} \dd{u})\\
            &= \frac{3}{2} \int_{0}^{1} \qty(1 - u)^{\frac{3}{2}} u^{\frac{1}{2}} \dd{u}\\
            &= \frac{3}{2} \int_{0}^{1} u^{\frac{3}{2}-1} \qty(1 - u)^{\frac{5}{2}-1}  \dd{u}\\
        \intertext{which is of the form of the beta function as defined in Definition \ref{def:beta_fun}}
            &= \frac{3}{2} B(\frac{3}{2}, \frac{5}{2})\\
            &= \frac{3}{2} \cfrac{
                                        \Gamma(\frac{3}{2})\Gamma(\frac{5}{2})
                                    }{
                                        \Gamma(\frac{3}{2} + \frac{5}{2})
                                    }\\
            &= \cfrac{
                3 \Gamma(\frac{3}{2})\Gamma(\frac{5}{2})
            }{
                2 \Gamma(4)
            }\\
            &= \cfrac{
                3 \qty(\frac{\sqrt{\pi}}{2}) \qty(\frac{3\sqrt{\pi}}{4})
            }{
                2 \qty(3!)
            }\\
            &= \cfrac{
                \frac{9 \pi}{8}
            }{
                (2)(3)(2)(1)
            }\\
            &= \frac{9 \pi}{96}\\
        \Aboxed{
            F(x) &= \frac{3 \pi}{32} \approx 0.29452
        }
    \end{align*}

\end{example}


% Problem 4 -------------------------------------------
\newpage
\section{}
Let \[
    f(x) = \begin{cases}
        x \sin(\frac{1}{x}) &\text{if } 0<x \leq 1;\\
        0 &\text{if } x = 0
    \end{cases}
\] Show that $f$ is bounded and continuous on $[0,1]$, but not of bounded variation on $[0,1]$.


% Bounded Function
\begin{definition} \label{def:bounded_fun}
    $f : (a,b) \to \R$ is a \emph{\underline{bounded function}} iff \[
        \exists_{N \in \R} \st \forall_{x \in (a,b)} \abs{f(x)} < N
    \]
\end{definition}

\begin{definition} \label{def:cont_fun}
    $f: (S_1, d_1) \to (S_2, d_2)$ is a \underline{\emph{continuous function}} iff \[
        \forall_{x\in S_1} \forall_{\epsilon>0} \exists_{\delta(x,\epsilon)>0} \forall_{y\in S_1} : d_1(x,y) < \delta \implies d_2(f(x),f(y)) < \epsilon
    \]
\end{definition} 

\begin{definition} \label{def:partition_variation}
For function $f: [a,b] \to \R$ and partition $P = \qty{a = x_0 < x_1 < \cdots < x_n = b}$:
    \begin{enumerate}
        \item the \emph{\underline{variaton of $f$ over $P$}} is defined as\[
            V_{a}^{b}(f,P) = \sum_{i=1}^{n} \abs{f(x_i) - f(x_{i-1})}
        \]
        \item the \emph{\underline{variaton of $f$ from $a$ to $b$}} is defined as\[
            V_{a}^{b}(f) = \sup_{P} V_{a}^{b}(f,P)
        \]
        \item $f$ is considered of \emph{\underline{bounded variation}} on $[a,b]$ if $V_{a}^{b}(f)$ is finite.
        \item the \emph{\underline{family of functions of bounded variaton on $[a,b]$}} is denoted as $BV_{a}^{b}$.
    \end{enumerate}
\end{definition}

\begin{example}
    Let \[
        f(x) = \begin{cases}
            x \sin(\frac{1}{x}) & 0 < x \leq 1\\
            0 & x = 0
        \end{cases}
    \] 
    \subsection{$f(x)$ is bounded}
    \begin{proof}
        For $x \in \{0\}$, $f(x) = 0$. 
        For $x \in (0,1]$, \[
            f(x) = x \sin(\frac{1}{x}) \leq x (1) \leq 1
        \] which is bounded. 
        Therefore, $f(x)$ is bounded $\forall_{x \in [0,1]}$.
    \end{proof}

    \subsection{$f(x)$ is continuous}
    \begin{proof}
        For $x \in \{0\}$, $f(x) = 0$.
        This means that \[
            \lim_{x \to 0^{-}} f(x) = 0
        \]
        
        For $x \in (0,1]$, \[
            f(x) = x \sin(\frac{1}{x})
        \] which is continuous on $(0,1]$.
        Additionally, this results in \[
            \lim_{x\to 0^{+}} f(x) = \lim_{x\to 0^{+}} x \sin(\frac{1}{x}) = 0
        \]
        Therefore, $f(x)$ is continuous $\forall_{x \in [0,1]}$.
    \end{proof}

    \subsection{$f(x)$ is not of bounded variation on $[0,1]$.}
    \begin{proof}
        In order for $f$ to be of bounded variation on $[a,b]$, the variation \[
            V_{a}^{b}(f) = \sup_{P} V_{a}^{b} (f,P)
        \] must be finite. 
        The existence of this bound can be demonstrated with the following counter-example:
        Let \[
            \qty{a}_k := \qty{a_k = (2 k + 1) \frac{\pi}{2} \forall_{k = 0, \dots, N-1}}
        \] For size $N$, define partition of $[0,1]$ \[
            P_N = \qty{
                0 = x_0 = 0 < x_1 = \frac{1}{a_{N-1}} < x_2 = \frac{1}{a_{N-2}} 
                < \cdots < x_{n-1} = \frac{1}{a_0} < x_{n} = 1
            }
        \] which can be used to construct a sequence, but that's not the point.

        The variation $V_{a}^{b}(f, P_N)$ is finite only for bounded $N$. 
        i.e. $\exists_{0 < M_N < \infty}$ that bounds the variation $V_{a}^{b}(f,P_N)$ for a given $N$:
        \begin{align*}
            V_{a}^{b}(f, P_N) 
                &= \sum_{i=1}^{N} \abs{f(x_i) - f(x_i-1)}\\
                &= \abs{\frac{1}{a_{N-1}} \sin(a_{N-1})} 
                    + \sum_{i=2}^{N} \abs{
                        \frac{1}{a_{N-i}} \sin(a_{N-i}) - \frac{1}{a_{N-i-1}} \sin(a_{N-i-1})
                    }\\
                &= \abs{\frac{1}{a_{N-1}}} 
                    + \sum_{i=1}^{N} 
                        \abs{\frac{1}{a_{N-i}} - \frac{1}{a_{N-i-1}}} 
                        \abs{(1) - (-1)}\footnote{or $\abs{-1 - 1}$}\\
                &= \abs{\frac{1}{a_{N-1}}} 
                    + 2 \sum_{i=1}^{N} \abs{\frac{a_{N-i} - a_{N-i-1}}{a_{N-i} a_{N-i-1}}}
        \end{align*}

        However, this variation over $[a,b]$ is since the sum does not converge as $N \to \infty$.
    \end{proof}

\end{example}




% Problem 5 -------------------------------------------
\newpage
\section{}
Assume $f$ is differentiable on $[a,b]$ with $\abs{f'(x)} \leq M < \infty$ for $a\leq x\leq b$.
Show that $f$ is of bounded variation and $V_a^b (f) \leq M(b-a)$.
(Hint: Use Mean Value Theorem)

\begin{theorem} \label{thm:mean_val}
    \textbf{Mean Value Theorem:}
    Let $f : [a,b] \to \R$ be continuous on $[a,b]$ and differentiable on $(a,b)$. 
    There exists $c \in (a,b)$ such that\[
        f'(c) = \cfrac{f(b) - f(a)}{b-a}
    \]
\end{theorem}


\begin{theorem}
    Let $f : [a,b] \to \R$ be differentiable on $[a,b]$. 
    If the derivative is bounded $\exists_{M > 0} \st \forall_{x \in [a,b]} \abs{f'(x)} \leq M < \infty$, 
    then $f$ will have a bounded variation with $V_{a}^{b}(f) \leq M(b-a)$.
    \begin{proof}
        From Definition \ref{def:partition_variation}, we have the following:
        The variation of $f$ associated with $P$ is \[
            V_{a}^{b}(f,P) = \sum_{i=1}^{N} \abs{f(x_i) - f(x_{i-1})}
        \]
        In order for $f$ to be of bounded variation on $[a,b]$, the variation \[
            V_{a}^{b}(f) = \sup_{P} V_{a}^{b} (f,P)
        \] must be finite.
        
        We also refer to the principles underlying Theorem \ref{thm:mean_val}, which ultimately states that \[
            \exists_{c \in (a,b)} \st f'(c) = \cfrac{f(b) - f(a)}{b-a}
        \] which represents the mean of the the derivative overall.

        Since $f'(x)$ has a bound, $\abs{f'(x)} \leq M$, $V_{a}^{b}(f,P)$ for a given $P$ will also be bounded.
        \begin{align*}
            V_{a}^{b}(f,P) 
                &= \sum_{i=1}^{N} \abs{f(x_i) - f(x_{i-1})}\\
                &\leq \sum_{i=1}^{N} \abs{M (x_{i} - x{i-1})}\\
                &= \sum_{i=1}^{N} \abs{M} \abs{x_{i} - x{i-1}}\\
                &= M \sum_{i=1}^{N} x_{i} - x_{i-1}\\
                &= M \qty(x_1 - x_0 + x_2 - x_1 + \dots + x_{N-1} - x_{N-2} + x_{N} - x_{N-1})\\
                &= M \qty(x_1 - x_1 + x_2 - x_2 + \dots + x_{N-1} - x_{N-1} + x_{N} - x_0)\\
                &= M \qty(x_{N} - x_0)\\
                &= M \qty(b - a)
        \end{align*}
        This means that\[
            V_{a}^{b}(f,P) \leq M (b-a)
        \] forall partitions of $[a,b]$.
        Therefore, by definition, the variation of $f$ on $[a,b]$ is
        \[\boxed{
            V_{a}^{b}(f) \leq M (b-a)
        }\]
    \end{proof}
\end{theorem}

\end{document}
