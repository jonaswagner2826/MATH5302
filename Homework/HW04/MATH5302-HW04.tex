% Standard Article Definition
\documentclass[]{article}

% Page Formatting
\usepackage[margin=1in]{geometry}
\setlength\parindent{0pt}

% Graphics
\usepackage{graphicx}

% Math Packages
\usepackage{physics}
\usepackage{amsmath, amsfonts, amssymb, amsthm}
\usepackage{mathtools}

% Extra Packages
\usepackage{listings}
\usepackage{hyperref}

% Section Heading Settings
\usepackage{enumitem}
\renewcommand{\theenumi}{\alph{enumi}}
\renewcommand*{\thesection}{Problem \arabic{section}}
\renewcommand*{\thesubsection}{\alph{subsection})}
\renewcommand*{\thesubsubsection}{}%\quad \quad \roman{subsubsection})}

%Custom Commands
\newcommand{\Rel}{\mathcal{R}}
\newcommand{\R}{\mathbb{R}}
\newcommand{\C}{\mathbb{C}}
\newcommand{\N}{\mathbb{N}}
\newcommand{\Z}{\mathbb{Z}}
\newcommand{\Q}{\mathbb{Q}}

\newcommand{\toI}{\xrightarrow{\textsf{\tiny I}}}
\newcommand{\toS}{\xrightarrow{\textsf{\tiny S}}}
\newcommand{\toB}{\xrightarrow{\textsf{\tiny B}}}

\newcommand{\divisible}{ \ \vdots \ }
\newcommand{\st}{\ : \ }

% Theorem Definition
\newtheorem{definition}{Definition}
\newtheorem{assumption}{Assumption}
\newtheorem{theorem}{Theorem}
\newtheorem{lemma}{Lemma}
\newtheorem{proposition}{Proposition}
\newtheorem{example}{Example}


%opening
\title{MATH 5302 Elementary Analysis II - Homework 2}
\author{Jonas Wagner}
\date{2022, Febuary 9\textsuperscript{th}}

\begin{document}

\maketitle

% Problem 1 ----------------------------------------------
\section{}
Show that\[
    B(\alpha,\beta) = \int_0^1 x^{\alpha-1} (1-x)^{\beta-1} \dd{x}
\] is well-defined for $\alpha > 0$ and $\beta > 0$.

\begin{definition}
    The improper integral \[
        \int_{0}^{a} f(x) \dd{x}
    \] is \emph{\underline{well-defined}} iff\[
        \lim_{\epsilon \to 0} \int_{0}^{a} f(x) \dd{x}
    \] exists.
\end{definition}

\begin{definition}
    The beta function $B(\alpha, \beta)$ is defined as \[
        B(\alpha,\beta) = \int_0^1 x^{\alpha-1} (1-x)^{\beta-1} \dd{x}
            = \cfrac{\Gamma(\alpha)\Gamma(\beta)}{\Gamma(\alpha + \beta)}
    \] for $\alpha>0$ and $\beta>0$.
\end{definition}


% \begin{theorem}\label{thm:comp_test}
%     \textbf{Comparison Test:}
%     Let $f,g:[a,b) \to \R$ be two functions such that 
%     (i) $f(x)$ and $g(x)$ are integrable on $[a,A]\subset[a,b)$, for $a<A<b$;
%     (ii) There exists $a < M < b$ such that $0\leq f(x) \leq g(x)$ forall $x \in [M,b)$.
%     Then,\begin{enumerate}
%         \item If $\int_a^b g(x) \dd{x}$ converges then $\int_a^b f(x) \dd{x}$ also converges;
%         \item If $\int_a^b f(x) \dd{x}$ diverges then $\int_a^b f(x) \dd{x}$ also diverges.
%     \end{enumerate}
% \end{theorem}

\begin{theorem}\label{thm:limit_comp_test}
    \textbf{Limit Comparison Test:}
    Let $f,g:[a,b) \to \R$ be two functions such that 
    (i) $f(x)$ and $g(x)$ are integrable on $[a,A]\subset[a,b)$, for $a<A<b$;
    (ii) There exists $a \leq K \leq b$ such that $\lim_{x\to b^{-}} \frac{f(x)}{g(x)} = K$.
    Then,\begin{enumerate}
        \item If $0< K < \infty$, then $\int_a^b g(x) \dd{x}$ converges iff $\int_a^b f(x)$ converges.
        \item If $K = 0$, then $\int_a^b g(x)$ converges implies $\int_a^b f(x) \dd{x}$ converges.
        \item If $K \infty 0$, then $\int_a^b g(x)$ divergent implies $\int_a^b f(x) \dd{x}$ divergent.
    \end{enumerate}
\end{theorem}

% \begin{theorem}\label{thm:cauchy_criterion}
%     \textbf{Cauchy Criterion:}
%     Let $f:[a,b) \to \R$ be a function integrable on every $[a,A]\subset[a,b)$, for $a<A<b$.
%     Then the imporper integral $\int_a^b f(x) \dd{x}$ converges if and only if\[
%         \forall_{\eta> 0} \exists_{\epsilon \in (0, b-a)} \st 
%         A, B \in (b-\epsilon,b) 
%         \implies \abs{\int_{A}^{B} f(x) \dd{x}} < \eta
%     \]
% \end{theorem}

\begin{theorem}
    The improper integral that defines the beta function \[
        B(\alpha,\beta) = \int_0^1 x^{\alpha-1} (1-x)^{\beta-1} \dd{x}
    \] is well-defined for $\alpha > 0$ and $\beta > 0$. 
    \begin{proof}
        The integrand of $B(\alpha,\beta)$, \[
            b(\alpha,\beta) = x^{\alpha-1} (1-x)^{\beta - 1}
        \] is not strictly bounded $\forall_{\alpha,\beta>0}$, but this is not necessary for convergence. 
        $\forall{\alpha,\beta \in [0,\infty)}$ the $b(\alpha,\beta)$ is bounded. 
        This makes $B(\alpha,\beta)$ a proper integral which is therefore convergent.

        In the other case the integrand is not bounded, but the improper integral still converges. 
        $\forall_{\alpha \in (0,1)}$ then $b(\alpha,\beta)$ is unbounded at $x = 0$.
        Similarly, $\forall_{\beta \in (0,1)}$ then $b(\alpha,\beta)$ is unbounded at $x = 1$.        

        The beta function can instead be split up into two parts:\[
            B(\alpha,\beta) 
                = \int_{0}^{c} x^{\alpha-1} (1-x)^{\beta-1} \dd{x}
                + \int_{c}^{1} x^{\alpha-1} (1-x)^{\beta-1} \dd{x}
        \] where $c \in (0,1)$.

        For the first improper integral, $\int_{0}^{c} x^{\alpha-1} (1-x)^{\beta-1} \dd{x}$
        a discontinuity exists at $x = 0$ for $\alpha \in (0,1)$. 
        Using the Limit Comparison Test from Theorem \ref{thm:limit_comp_test} with $g(x) = x^{\alpha - 1}$, 
        \begin{align*}
            \lim_{x\to 0^{+}} \frac{f(x)}{g(x)} 
                &= \lim_{x\to 0^{+}} \frac{x^{\alpha-1} (1-x)^{\beta-1}}{x^{\alpha - 1}}\\
                &= \lim_{x\to 0^{+}} (1-x)^{\beta-1}\\
                &= 1 \neq 0
        \end{align*}
        Which then implies that $\int_{0}^{c} x^{\alpha-1} (1-x)^{\beta-1} \dd{x}$ 
        converges $\forall_{\alpha,\beta > 0}$.

        For the second improper integral, $\int_{c}^{1} x^{\alpha-1} (1-x)^{\beta-1} \dd{x}$
        a discontinuity exists at $x = 1$ for $\beta \in (0,1)$. 
        Using the Limit Comparison Test from Theorem \ref{thm:limit_comp_test} with $g(x) = (1-x)^{\beta - 1}$, 
        \begin{align*}
            \lim_{x\to 1^{-}} \frac{f(x)}{g(x)} 
                &= \lim_{x\to 1^{-}} \frac{x^{\alpha-1} (1-x)^{\beta-1}}{(1-x)^{\beta - 1}}\\
                &= \lim_{x\to 1^{-}} x^{\alpha - 1}\\
                &= 1 \neq 0
        \end{align*}
        Which then implies that $\int_{c}^{1} x^{\alpha-1} (1-x)^{\beta-1} \dd{x}$ 
        converges $\forall_{\alpha,\beta > 0}$.

        Together, the convergence of $\int_{0}^{c} x^{\alpha-1} (1-x)^{\beta-1} \dd{x}$ 
        and $\int_{c}^{1} x^{\alpha-1} (1-x)^{\beta-1} \dd{x}$ 
        implies that $B(\alpha,\beta)$ converges $\forall_{\alpha,\beta>0}$ and therefore $B(\alpha,\beta)$ is well defined.
    \end{proof}
\end{theorem}



% Problem 2 -------------------------------------------
\newpage
\section{}
Show that $f$ if Riemann integrable on $[a,b]$, then \[
    \lim_{\epsilon\to 0^{+}} \int_{a}^{b - \epsilon} f(x) \dd{x} = \int_{a}^{b} f(x) \dd{x}
\]









% Problem 3 -------------------------------------------
\newpage
\section{}
Evaluate $\int_0^1 (1-x^{\frac{2}{3}})^{\frac{3}{2}} \dd{x}$.
Hint: Express the integral in terms of the gamma function first.





% Problem 4 -------------------------------------------
\newpage
\section{}
Let \[
    f(x) = \begin{cases}
        x \sin(\frac{1}{x}) &\text{if } 0<x \leq 1;\\
        0 &\text{if } x = 0
    \end{cases}
\] Show that $f$ is bounded and continuous on $[0,1]$, but not of bounded variation on $[0,1]$.





% Problem 5 -------------------------------------------
\newpage
\section{}
Assume $f$ is differentiable on $[a,b]$ with $\abs{f'(x)} \leq M < \infty$ for $a\leq x\leq b$.
Show that $f$ is of bounded variation and $V_a^b (f) \leq M(b-a)$.
(Hint: Use Mean Value Theorem)







\end{document}
