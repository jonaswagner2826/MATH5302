% Standard Article Definition
\documentclass[]{article}

% Page Formatting
\usepackage[margin=1in]{geometry}
\setlength\parindent{0pt}

% Graphics
\usepackage{graphicx}

% Math Packages
\usepackage{physics}
\usepackage{amsmath, amsfonts, amssymb, amsthm}
\usepackage{mathtools}

% Extra Packages
\usepackage{listings}
\usepackage{hyperref}

% Section Heading Settings
\usepackage{enumitem}
\renewcommand{\theenumi}{\alph{enumi}}
\renewcommand*{\thesection}{Problem \arabic{section}}
\renewcommand*{\thesubsection}{\alph{subsection})}
\renewcommand*{\thesubsubsection}{}%\quad \quad \roman{subsubsection})}

%Custom Commands
\newcommand{\Rel}{\mathcal{R}}
\newcommand{\R}{\mathbb{R}}
\newcommand{\C}{\mathbb{C}}
\newcommand{\N}{\mathbb{N}}
\newcommand{\Z}{\mathbb{Z}}
\newcommand{\Q}{\mathbb{Q}}

\newcommand{\toI}{\xrightarrow{\textsf{\tiny I}}}
\newcommand{\toS}{\xrightarrow{\textsf{\tiny S}}}
\newcommand{\toB}{\xrightarrow{\textsf{\tiny B}}}

\newcommand{\divisible}{ \ \vdots \ }
\newcommand{\st}{\ : \ }

% Theorem Definition
\newtheorem{definition}{Definition}
\newtheorem{assumption}{Assumption}
\newtheorem{theorem}{Theorem}
\newtheorem{lemma}{Lemma}
\newtheorem{proposition}{Proposition}
\newtheorem{example}{Example}


%opening
\title{MATH 5302 Elementary Analysis II - Homework 5}
\author{Jonas Wagner}
\date{2022, March 28\textsuperscript{th}}

\begin{document}

\maketitle


\section*{Preliminaries}
\begin{definition}
    \textbf{\emph{Darboux-Stieltjes Integral}}
    Let $f : [a,b] \to \R$ and $\alpha : [a,b] \to \R$, with $f$ bounded and $\alpha$ increasing on $[a,b]$.
    Let partition $P$ be defined as \[
        P = \qty{a = x_0 < x_1 < \cdots < x_{n-1} < x_n = b}
    \] Let \[
        M(f, [x_1, x_2]) = \sup_{x \in [x_1, x_2]} f(x)
    \] and \[
        m(f, [x_1, x_2]) = \inf_{x \in [x_1, x_2]} f(x)
    \]
    \begin{enumerate}
        \item The upper and lower \emph{\underline{Darboux-Stieltjes Sums}} are defined \[            
            U(f, \alpha, P) = \sum_{i=1}^{n} M(f, [x_{i-1}, x_i]) \cdot \Delta_i \alpha
        \] and \[
            L(f, \alpha, P) = \sum_{i=1}^{n} m(f, [x_{i-1}, x_i]) \cdot \Delta_i \alpha
        \] respectively with \[
            \Delta_i \alpha = \alpha(x_i) - \alpha(x_{i-1})
        \] A more general sum $S(f, \alpha, P)$ is when $f(x_i^*)$ for $x_i^* \in [x_{i-1}, x_{i}]$ is used instead.

        \emph{\textbf{\underline{Note:}}} \[
            m(f,[a,b]) \cdot (\alpha(b) - \alpha(a)) 
            \leq L(f, \alpha, P) 
            \leq U(f, \alpha, P)
            \leq M(f,[a,b]) \cdot (\alpha(b) - \alpha(a)) 
        \]
        \item The upper and lower \emph{\underline{Darboux-Stieltjes Integrals}} are defined \[            
            U(f, \alpha) = \inf_{P \text{ partition of } [a,b]} U(f, \alpha, P)
        \] and \[
            L(f, \alpha) = \sup_{P \text{ partition of } [a,b]} U(f, \alpha, P)
        \] respectively.

        \emph{\textbf{\underline{Note:}}}\[
            L(f, \alpha)
            \leq L(f, \alpha, P) 
            \leq U(f, \alpha, P)
            \leq U(f, \alpha)
        \] for any $P$ partition of $[a,b]$.
        \item $f$ is called \emph{\underline{Darboux-Stieltjes Integrable}} with respect to $\alpha$ if and only if \[
            \forall_{\epsilon>0} \exists_{P = \qty{a = x_0 < x_1 < \cdots < x_{n-1} < x_n = b}} : 
            U(f,\alpha,P) - L(f,\alpha,P) < \epsilon
        \] in which case the \emph{\underline{Darboux-Stieltjes Integral}} with respect to $\alpha$ is defined as \[
            \mathcal{DS} \int_{a}^{b} f \dd{\alpha} = U(f,\alpha) = L(f, \alpha)
        \]
        \emph{\textbf{\underline{Note:}}}
        If $f$ is also continuous on $[a,b]$ then $f$ is Riemann-Stieljes integrable which implies $f$ is Darboux-Stieljes integrable.
    \end{enumerate}
    \textbf{\emph{\underline{Properties:}}}
    When $f$ is Darboux-Stieltjes integrable on $[a,b]$ and $\alpha$ is increasing on $[a,b]$ then
    \begin{enumerate}
        \item $\abs{f}$ is Darboux-Stieltjes integrable on $[a,b]$ and \[
            \mathcal{DS} \int_{a}^{b} f \dd{\alpha} \leq \mathcal{DS} \int_{a}^{b} \abs{f} \dd{\alpha}
        \]
        \item $f^2$ is Darboux-Stieltjes integrable on $[a,b]$.
        \item If $g$ is also Darboux-Stieltjes integrable on $[a,b]$, then $fg$ is Darboux-Stieltjes integrable on $[a,b]$.
        \item For $\alpha_1$ and $\alpha_2$ also increasing on $[a,b]$ and $f$ is Darboux-Stieltjes integrable with respect to $\alpha_1$ and $\alpha_2$, then $f$ is Darboux-Stieltjes integrable with respect to $\alpha_1$ and $\alpha_2$. 
        Additionally, \begin{align*}
            & \mathcal{DS} \int_{a}^{b} f(x) \dd{\alpha_1(x)} + \mathcal{DS} \int_{a}^{b} f(x) \dd{\alpha_2(x)}\\
            = & \mathcal{DS} \int_{a}^{b} f(x) \dd{\alpha(x) + \alpha_2(x)}
        \end{align*}
        \item For $a < c < b$, $f$ is Darboux-Stieltjes integrable with respect to $\alpha$ on $[a,b]$ if and only if $f$ is Darboux-Stieltjes integrable with respect to $\alpha$ on $[a,c]$ and $[c,b]$. 
        Furthermore, \[
            \mathcal{DS} \int_{a}^{b} f(x) \dd{\alpha(x)}
            = \mathcal{DS} \int_{a}^{c} f(x) \dd{\alpha(x)} 
            + \mathcal{DS} \int_{c}^{b} f(x) \dd{\alpha(x)}
        \]
    \end{enumerate}
\end{definition}

\begin{definition}
    \textbf{Continuity:}
    Let $f : [a,b] \to \R$. 
    \begin{enumerate}
        \item $f$ is \emph{\underline{Lipschitz Continuous}} on $[a,b]$ if \[
            \exists_{C} : \forall_{x,y \in [a,b]} \abs{f(x) - f(y)} \leq \abs{x - y} 
        \]
        \item $f$ is \emph{\underline{Absolutely Continuous}} on $[a,b]$ if \[
            \forall_{\epsilon>0} \exists_{\delta>0}
            \forall_{
                \text{finite collection } \qty{(x,x')} \text{ of nonoverlaping intervals} 
                : \sum_{i=1}^{n} \abs{x_i' - x_i} < \delta
            } \sum_{i=1}^{n} \abs{f(x_i') - f(x_i)} < \epsilon
        \]
        \item $f$ is \underline{\emph{uniformly continuous}} on $[a,b]$ if \[
            \forall_{\epsilon>0} \exists_{\delta>0} : 
                \qty(x,y \in [a,b]) \land \qty{\abs{x-y} < \delta}
            \implies \abs{f(x) - f(y)} < \epsilon
        \]
        \item $f$ is \emph{\underline{continuous}} on $[a,b]$ if $f$ is continuous at all $x_0 \in [a,b]$.
        i.e. \[
            \forall_{\epsilon > 0} \exists_{\delta>0} : 
            \forall_{x \in [a,b]} \land \abs{x - x_0} < \delta
            \implies \abs{f(x) - f(x_0)} < \epsilon
        \]
    \end{enumerate}
    \emph{\underline{\textbf{Properties:}}}
    \begin{enumerate}
        \item $f$ continuous on closed $[a,b]$, then $f$ is uniformly continuous on $[a,b]$.
        \item $f$ differentiable at $x \in [a,b]$ implies Locally Lipschitz continuous at $x$.
        \item $C^1[a,b]$ is the set of differentiable functions with continuous derivatives on $[a,b]$.
        \item $C^1[a,b]$ $\subset$ differentiable functions with bounded derivatives
        \item Differentiable with bounded derivatives 
        $\implies$ Lipschitz continuous 
        $\implies$ Absolutely continuous
        $\implies$ uniformly continuous
        $\implies$ continuous
    \end{enumerate}
\end{definition}



% Problem 1 ----------------------------------------------
\newpage
\section{}
Let $f$ be a real-valued bounded function on $[-1,1]$. 
Let\[
    \alpha(x) = \begin{cases}
        0 &\text{if} \ -1 \leq x < 0;\\
        2 &\text{if} \  0 \leq x \leq 1.
    \end{cases}
\]
Assume $f$ is Riemann-Stieljes integrable with respect to $\alpha$ on $[-1,1]$. 
Show that \begin{enumerate}
    \item $f$ is continuous at $0$ from the left.
    \item $\int_{-1}^{1} f(x) \dd{\alpha(x)} = 2 f(0)$.
\end{enumerate}

% Part a
\subsection{$f$ is continuous at $0$ from the left}
\begin{example}
    Let $f : [-1, 1] \to \R$ bounded.
    Let \[
        \alpha(x) = \begin{cases}
            0   & -1 \leq x < 0\\
            2   &  0 \leq x < 1
        \end{cases}
    \] If $f$ is Riemann-Stieljes integrable w.r.t. $\alpha$ on $[-1,1]$, then $f$ is continuous at 0 from the left.
    \begin{proof}
        $f$ is Riemann-Stieljes integrable w.r.t. $\alpha$ on $[-1,1]$ means that \[
            \forall_{\epsilon > 0} \exists_{\gamma}  \exists_{\delta > 0} :
            \forall_{P : \text{mesh}(P) < \delta} \implies \abs{S(f, \alpha, P) - \epsilon}
        \] For \[
            P = \qty{-1 = x_0 < x_1 < \cdots < x_{k-1} = - \gamma < x_{k} = 0 < x_{k-1} < \cdots x_n = 1}
        \] with $\text{mesh}(P) < \delta$.
        For $x$ s.t. $- \gamma < x < 0$
        Select $x_i^* \in [x_{i}, x_{i+1}]$ and $x_{k-1}^* = x$. \begin{align*}
            \sum_{i=1}^{n} f(x_{i}^*) \Delta_i \alpha 
                &= f(x) (\alpha(x_k) - \alpha(x_{k-1}))\\
                &= f(x) (2 - 0) = 2 f(x)\\
            \sum_{i=1}^{n} f(x_{i}^*) \Delta_i \alpha 
                &= 2 f(x)
        \end{align*} Therefore, \[
            \abs{f(x) - \gamma} < \epsilon
        \] and \[
            \lim_{x \to 0^-} f(x) = \gamma
        \] Then by taking $x_{k-1}^* = 0$, the Riemann-Stieltjes sum $S = f(0)$, therefore \[
            \abs{f(x = 0) - \gamma} < \epsilon, \ \forall_{\epsilon > 0}
        \] which is the definition of continuity meaning that $f$ is continuous at $0$ from the left.
    \end{proof}
\end{example}

% Part b
\subsection{$\int_{-1}^{1} f(x) \dd{\alpha(x)} = 2 f(0)$}
\begin{example}
    Let $f : [-1, 1] \to \R$ bounded.
    Let \[
        \alpha(x) = \begin{cases}
            0   & -1 \leq x < 0\\
            2   &  0 \leq x < 1
        \end{cases}
    \] If $f$ is Riemann-Stieljes integrable w.r.t. $\alpha$ on $[-1,1]$, then $\int_{-1}^{1} f(x) \dd{\alpha(x)} = 2 f(0)$
    \begin{proof}
        Let \[
            P = \qty{-1 = x_0 < x_1 < \cdots < x_{k-1} = - \gamma < x_{k} = 0 < x_{k-1} < \cdots x_n = 1}
        \] with $\text{mesh}(P) < \delta$. 
        By definition, \[
            \mathcal{DS} \int_{a}^{b} f(x) \dd{\alpha(x)} = \lim_{\delta \to 0} S(f, \alpha, P)
        \] and \begin{align*}
            S(f, \alpha, P) 
                &= \sum_{i=1}^{n} f(x_{i}^*) \Delta_i \alpha\\
                &= \sum_{i=1}^{n} f(x_{i}^*) (\alpha(x_i) - \alpha(x_{i-1}))
            \intertext{since $\alpha(x_i) - \alpha(x_{i-1}) = 0 \forall_{i \neq k}$}
                &= f(x_{k}) (\alpha(x_k) - \alpha(x_{k-1}))\\
                &= f(x_k) (2 - 0) = 2 f(x_k) = 2 f(0)\\
            \Aboxed{
                \mathcal{DS} \int_{a}^{b} f(x) \dd{\alpha(x)} 
                % = S(f, \alpha, P) \forall_{P}
                &= 2 f(0)
            }
        \end{align*}
    \end{proof}
\end{example}


% Problem 2 ----------------------------------------------
\newpage
\section{}
Let $f$ and $\alpha$ be real-valued bounded functions on $[a,b]$ and $\alpha$ is increasing. 
Let $L(f,\alpha)$ and $U(f,\alpha)$ represents the lower and upper Darboux-Stieltjes integral of $f$ with respect to $\alpha$ on $[a,b]$, respectively,\begin{enumerate}
    \item Show that $U(f,\alpha) \leq U(\abs{f},\alpha)$.
    \item Is it true that $L(f,\alpha) \leq L(\abs{f},\alpha)$?
\end{enumerate}

\begin{example}
    Let $f : [a,b] \to \R$ and $\alpha : [a,b] \to \R$, with $f$ and $\alpha$ both bounded with $\alpha$ also increasing on $[a,b]$.
    Let partition $P$ be defined as \[
        P = \qty{a = x_0 < x_1 < \cdots < x_{n-1} < x_n = b}
    \] Let \[
        M(f, [x_1, x_2]) = \sup_{x \in [x_1, x_2]} f(x)
    \] and \[
        m(f, [x_1, x_2]) = \inf_{x \in [x_1, x_2]} f(x)
    \] Then by definition, \[            
        U(f, \alpha, P) = \sum_{i=1}^{n} M(f, [x_{i-1}, x_i]) \cdot \Delta_i \alpha
    \] and \[
        L(f, \alpha, P) = \sum_{i=1}^{n} m(f, [x_{i-1}, x_i]) \cdot \Delta_i \alpha
    \] respectively with \[
        \Delta_i \alpha = \alpha(x_i) - \alpha(x_{i-1})
    \] Additionally, \[            
        U(f, \alpha) = \inf_{P \text{ partition of } [a,b]} U(f, \alpha, P)
    \] and \[
        L(f, \alpha) = \sup_{P \text{ partition of } [a,b]} U(f, \alpha, P)
    \]
    \begin{enumerate}
        \item \emph{\underline{\textbf{Show that $U(f,\alpha) \leq U(\abs{f},\alpha)$.}}} \begin{proof}
            Since $f$ is a real-valued function, $\forall_{x \in [a,b]} f(x) \leq \abs{f(x)}$. 
            For every $[x_1,x_2] \in [a,b]$, \[
                M(f, [x_1, x_2]) 
                    = \sup_{x \in [x_1, x_2]} f(x) 
                    \leq \sup_{x \in [x_1, x_2]} \abs{f(x)} 
                = M(\abs{f}, [x_1, x_2])
            \] Then for every $P$, \[
                U(f, \alpha, P) 
                    = \sum_{i=1}^{n} M(f, [x_{i-1}, x_i]) \cdot \Delta_i \alpha 
                    \leq \sum_{i=1}^{n} M(\abs{f}, [x_{i-1}, x_i]) \cdot \Delta_i \alpha 
                = U(\abs{f}, \alpha, P)
            \] Therefore, \[
                U(f, \alpha) 
                    = \inf_{P \text{ partition of } [a,b]} U(f, \alpha, P)
                    \leq \inf_{P \text{ partition of } [a,b]} U(\abs{f}, \alpha, P)
                = U(\abs{f}, \alpha)
            \]
        \end{proof}
        \item \emph{\textbf{\underline{Is it true that $L(f,\alpha) \leq L(\abs{f},\alpha)$?}}} 
        Yes, by the same logic as the previous statement. 
        Essentially since every value of $f \leq \abs{f}$, the same progression is true.
        \begin{proof}
            Since $f$ is a real-valued function, $\forall_{x \in [a,b]} f(x) \leq \abs{f(x)}$. 
            For every $[x_1,x_2] \in [a,b]$, \[
                m(f, [x_1, x_2]) 
                    = \inf_{x \in [x_1, x_2]} f(x) 
                    \leq \inf_{x \in [x_1, x_2]} \abs{f(x)} 
                = m(\abs{f}, [x_1, x_2])
            \] Then for every $P$, \[
                L(f, \alpha, P) 
                    = \sum_{i=1}^{n} m(f, [x_{i-1}, x_i]) \cdot \Delta_i \alpha 
                    \leq \sum_{i=1}^{n} m(\abs{f}, [x_{i-1}, x_i]) \cdot \Delta_i \alpha 
                = L(\abs{f}, \alpha, P)
            \] Therefore, \[
                L(f, \alpha) 
                    = \sup_{P \text{ partition of } [a,b]} L(f, \alpha, P)
                    \leq \sup_{P \text{ partition of } [a,b]} L(\abs{f}, \alpha, P)
                = L(\abs{f}, \alpha)
            \]
        \end{proof}
    \end{enumerate}        
\end{example}

% Problem 3 ----------------------------------------------
\newpage
\section{}
Let $\alpha$ be a bounded real-valued increasing function on $[a,b]$. 
Assume $a < c < b$ and $\alpha$ is continuous at $c$. 
Let \[
    f(x) = \begin{cases}
        1   &\text{if } x = c;\\
        0   &\text{if } x \neq c.
    \end{cases}
\] Show directly that $f$ is Darboux-Stieltjes integrable on $[a,b]$ and $\int_{a}^{b} f(x) \dd{\alpha(x)} = 0$. 
(Do not use Theorem 8.16.)

\begin{example}
    Let $\alpha : [a,b] \to \R$ bounded and increasing.
    Let $c \in (a,b)$ such that $\alpha$ is continuous at $c$.
    Let \[
        f(x) = \begin{cases}
            1   &\text{if } x = c;\\
            0   &\text{if } x \neq c.
        \end{cases}
    \]
    \begin{enumerate}
        \item \textbf{\emph{\underline{$f$ is Darboux-Stieltjes integrable on $[a,b]$.}}}
        \begin{proof}
            By definition, $f$ is Darboux-Stieltjes integrable with respect to $\alpha$ on $[a,b]$ if and only if \[
                \forall_{\epsilon>0} \exists_{P = \qty{a = x_0 < x_1 < \cdots < x_{n-1} < x_n = b}} : 
                U(f,\alpha,P) - L(f,\alpha,P) < \epsilon
            \] 
            Let $\epsilon > 0$. 
            Let partition $P$ be defined as \[
                P = \qty{a = x_0 < x_1 < \cdots < x_{k-1} < x_{k} = c < x_{k-1} < \cdots < x_{n-1} < x_n = b}
            \] such that $\text{mesh}(P) < \delta$ for some $\delta > 0$.
            % Select $x_i^* \in [x_{i-1},x_i]$ s.t. $f(x_i^*) < \inf_{x \in [x_{i-1}, x_i]} f(x) + \epsilon$
            The lower Darboux-Stieltjes sum is given by \begin{align*}
                L(f, \alpha, P) 
                    &= \sum_{i=1}^n m(f, [x_{i-1},x_{i}]) \Delta_i \alpha\\
                    &= \sum_{i=1}^n (\inf_{x \in [x_{i-1}, x_i]} f(x)) \Delta_i \alpha\\
                \intertext{since $f(x) = 0 \forall_{i} \neq k$,}
                    &= \sum_{i=1}^{k-1} (0) \Delta_i \alpha
                        + \sum_{i=k+2}^{n} (0) \Delta_i \alpha
                        + \inf_{x \in [x_{k-1}, x_k]} f(x) \Delta_k \alpha
                        + \inf_{x \in [x_{k}, x_{k-1}]} f(x) \Delta_{k+1} \alpha
                \intertext{which still results in $0$ terms}
                    &= 0 + 0 + 0 (\Delta_k \alpha) + 0 (\Delta_{k+1} \alpha)\\
                L(f, \alpha, P) &= 0
            \end{align*}

            The upper Darboux-Stieltjes sum is given by \begin{align*}
                U(f, \alpha, P) 
                    &= \sum_{i=1}^n M(f, [x_{i-1},x_{i}]) \Delta_i \alpha\\
                    &= \sum_{i=1}^n (\sup_{x \in [x_{i-1}, x_i]} f(x)) \Delta_i \alpha\\
                \intertext{since $f(x) = 0 \forall_{i} \neq k$,}
                    &= \sum_{i=1}^{k-1} (0) \Delta_i \alpha\\
                        &\quad + \sum_{i=k+2}^{n} (0) \Delta_i \alpha\\
                        &\quad + \sup_{x \in [x_{k-1}, x_k]} f(x) \Delta_k \alpha\\
                        &\quad + \sup_{x \in [x_{k}, x_{k-1}]} f(x) \Delta_{k+1} \alpha\\
                    &= 0 + 0 + 1 (\Delta_k \alpha) + 1 (\Delta_{k+1} \alpha)\\
                    &= (\alpha(x_{k}) - \alpha(x_{k-1})) + (\alpha(x_{k+1}) - \alpha(x_{k}))\\
                    &= \alpha(x_{k+1}) - \alpha(x_{k-1}\\
                \intertext{since $\text{mesh}(P) < \delta$,}
                    &\leq 2 \delta\\
                U(f, \alpha, P) &\leq 2 \delta
            \end{align*}
            
            Thus, setting $\delta < \frac{\epsilon}{2}$ results in \[
                U(f, \alpha, P) - L(f, \alpha, P) \leq 2\delta < \epsilon
            \]
        \end{proof}
        \item \emph{\underline{\textbf{$\int_{a}^{b} f(x) \dd{\alpha(x)} = 0$}}}
        \begin{proof}
            From above we have $L(f, \alpha, P) = 0$ and $U(f, \alpha, P) \leq 2 \delta$ for $\text{mesh}(P) < \delta$.
            So we have $L(f, \alpha) = \sup_{P} L(f, \alpha, P) = 0$.
            Taking the limit as $\delta \to 0$, we have \[
                U(f, \alpha) = \lim_{\delta \to 0} U(f, \alpha, P) \leq \lim_{\delta \to 0} 2 \delta = 0
            \] so \[
                \mathcal{DS} \int_{a}^{b} f \dd{\alpha} = U(f,\alpha) = L(f, \alpha) = 0
            \]
        \end{proof}
    \end{enumerate}
\end{example}

% Problem 4 ----------------------------------------------
\newpage
\section{}
Let $f$ and $\alpha$ be real-valued bounded functions on $[a,b]$ and $\alpha$ is increasing on $[a,b]$. 
Assume $f$ is Darboux-Stieltjes integrable with respect to $\alpha$ on $[a,b]$. 
Let $[c,d] \subset [a,b]$. 
Show that $f$ is Darboux-Stieltjes integrable with respect to $\alpha$ on $[c,d]$.

\begin{example}
    Let $f : [a,b] \to \R$ and $\alpha : [a,b] \to \R$, with $f$ bounded on $[a,b]$ and $\alpha$ bounded and increasing on $[a,b]$.
    Let partition $P$ be defined as \[
        P = \qty{a = x_0 < x_1 < \cdots < x_{n-1} < x_n = b}
    \] Assume $f$ is Darboux-Stieltjes integrable with respect to $\alpha$ on $[a,b]$. 
    
    $f$ is is Darboux-Stieltjes integrable with respect to $\alpha$ on all $[c,d] \subset [a,b]$
    \begin{proof}
        $f$ is Darboux-Stieltjes integrable with respect to $\alpha$ on $[a,b]$ means that \[
            \forall_{\epsilon>0} \exists_{P = \qty{a = x_0 < x_1 < \cdots < x_{n-1} < x_n = b}} : 
            U(f,\alpha,P) - L(f,\alpha,P) < \epsilon
        % \] which means $\exists_{P}$ resulting in \[
        %     U(f,\alpha, P) - L(f,\alpha, P) = 0
        \] when taking the limit of epsilon to 0. 
        For this $P$, find $cl$, $cu$, $dl$, and $du$ so that\[
            P = \qty{a = x_0 < x_1 < \cdots < x_{cl} < c < x_{cu} < \cdots < x_{dl} < d < x_{du} < \cdots < x_n = b}
        \] Take \[
            P_{cd} = \qty{x_{cl} < c < x_{c} < \cdots < x_{dl} = d < x_{du}}
        \]
        \begin{align*}
            U(f,\alpha,P) - L(f,\alpha,P) 
                &= \sum_{i=1}^n M(f, [x_{i-1},x_{i}]) \Delta_i \alpha\\
                    &\quad - \sum_{i=1}^n m(f, [x_{i-1},x_{i}]) \Delta_i \alpha
                    &&< \epsilon\\
                &= \sum_{i=1}^{cl-1} M(f, [x_{i-1},x_{i}]) \Delta_i \alpha\\
                    &\quad + \sum_{i=cl}^{du} M(f, [x_{i-1},x_{i}]) \Delta_i \alpha\\
                    &\quad + \sum_{i=du+1}^n M(f, [x_{i-1},x_{i}]) \Delta_i \alpha\\
                    &\quad - \sum_{i=1}^{cl-1} m(f, [x_{i-1},x_{i}]) \Delta_i \alpha\\
                    &\quad - \sum_{i=cl}^{du} m(f, [x_{i-1},x_{i}]) \Delta_i \alpha\\
                    &\quad - \sum_{i=du+1}^{n} m(f, [x_{i-1},x_{i}]) \Delta_i \alpha
                    &&< \epsilon\\
                % &= \sum_{i=1}^{cl} M(f, [x_{i-1},x_{i}]) \Delta_i \alpha\\
                %     &\quad + \sum_{i=cu}^{dl} M(f, [x_{i-1},x_{i}]) \Delta_i \alpha\\
                %     &\quad + \sum_{i=du}^n M(f, [x_{i-1},x_{i}]) \Delta_i \alpha\\
                %     &\quad - \sum_{i=1}^{cl} m(f, [x_{i-1},x_{i}]) \Delta_i \alpha\\
                %     &\quad - \sum_{i=cu}^{dl} m(f, [x_{i-1},x_{i}]) \Delta_i \alpha\\
                %     &\quad - \sum_{i=du}^{n} m(f, [x_{i-1},x_{i}]) \Delta_i \alpha
                %     &&< \epsilon
        \end{align*}
        Observing the bounded sum \[
            \sum_{i=cl}^{du} M(f, [x_{i-1},x_{i}]) \Delta_i \alpha 
            - \sum_{i=cl}^{du} m(f, [x_{i-1},x_{i}]) \Delta_i \alpha
            < \epsilon - (***)
        % \]
        % and \[
        %     \sum_{i=cu}^{dl} M(f, [x_{i-1},x_{i}]) \Delta_i \alpha 
        %     - \sum_{i=cu}^{dl} m(f, [x_{i-1},x_{i}]) \Delta_i \alpha
        %     < \epsilon_2
        \] with $***$ being all the other bounded terms transferred to the RHS.
        This is a sum that bounds the $U(f,\alpha,P_{cd}) - L(f,\alpha,P_{cd})$ from above by potentially including the regions $[x_{cl}, c)$ and $(d, x_{du}]$ which would have an additional bounded term.
        Taking $\text{mesh}(P) \to 0$ we cause $x_{cl} \to c$ and $x_{du} \to d$ so that \[
            U(f,\alpha,P_{cd}) - L(f,\alpha,P_{cd}) 
                = \sum_{i=cl}^{du} M(f, [x_{i-1},x_{i}]) - m(f, [x_{i-1}, x_{i}]) \Delta_i \alpha
                < \epsilon_{cd} = \epsilon - (***)
        \] and therefore since $\epsilon - (***)$ is bounded for all $\delta$, \[
            \forall_{\epsilon>0} \exists_{\delta} : \text{mesh}(P) < \delta \implies U(f,\alpha,P_{cd}) - L(f,\alpha,P_{cd}) < \epsilon_{cd}
        \]
    \end{proof}
\end{example}

% Problem 5 ----------------------------------------------
\newpage
\section{}
Let $\alpha$ be a real-valued bounded function on $[a,b]$ and $\alpha$ is increasing with $\alpha(a) < \alpha(b)$. 
Let \[
    f(x) = \begin{cases}
        1   &\text{if $x$ is rational};\\
        0   &\text{if $x$ is irrational}.\\
    \end{cases}
\] Show that if $\alpha$ is continuous on $[a,b]$, then $f$ is not Darboux-Stieltjes integrable with respect to $\alpha$ on $[a,b]$.

\begin{example}
    Let $\alpha : [a,b] \to \R$ bounded and increasing with $\alpha(a) < \alpha(b)$.
    Let \[
        f(x) = \begin{cases}
            1   &\text{if $x$ is rational};\\
            0   &\text{if $x$ is irrational}.\\
        \end{cases}
    \] If $\alpha$ is continuous on $[a,b]$ then $f$ is not Darboux-Stieltjes integrable with respect to $\alpha$ on $[a,b]$.
    \begin{proof}
        By definition, $\alpha$ continuous on the closed interval $[a,b]$ implies $\alpha$ is uniformly continuous on $[a,b]$. 
        This means that \[ 
            \forall_{\epsilon_1} \exists_{\delta>0} : a \leq x < y \leq b, \ y - x < \delta 
            \implies \alpha(y) - \alpha(x) < \epsilon_1
        % \]
        % $\alpha$ is continuous at every $x_0$, where $\alpha$ is continuous at $x_0$ iff \[
        %     \forall_{\epsilon > 0} \exists_{\delta>0} : 
        %     \forall_{x \in [a,b]} \land \abs{x - x_0} < \delta
        %     \implies \abs{\alpha(x) - \alpha(x_0)} < \epsilon
        \] In order for $f$ to be Darboux-Stieltjes integrable with respect to $\alpha$ on $[a,b]$, then for every $\epsilon_2$ there must exists a partition $P$ of $[a,b]$, \[
            P = \qty{a = x_0 < x_1 < \cdots < x_n = b}
        \] so that \[
            U(f, \alpha, P) - L(f, \alpha, P) < \epsilon_2
        \] For this example, we have \begin{align*}
            U(f,\alpha,P) - L(f,\alpha,P) 
                &= \sum_{i=1}^n M(f, [x_{i-1},x_{i}]) \Delta_i \alpha 
                    - \sum_{i=1}^n m(f, [x_{i-1},x_{i}]) \Delta_i \alpha\\
            \intertext{Since $\alpha$ is continuous and increasing, $\Delta_i \alpha = \alpha(x_{i-1})-\alpha(x_{i}) > 0$}
                &= \sum_{i=1}^n M(f, [x_{i-1},x_{i}]) \qty(\alpha(x_{i-1})-\alpha(x_{i})) 
                    - \sum_{i=1}^n m(f, [x_{i-1},x_{i}]) \qty(\alpha(x_{i-1})-\alpha(x_{i}))
            \intertext{For $f$ we have that $M(f, [x_1, x_2]) = 1$ and $m(f, [x_1, x_2]) = 0$ $\forall_{x_1 \neq x_2 \in [a,b]}$}
                &= \sum_{i=1}^n (1) \qty(\alpha(x_{i-1})-\alpha(x_{i})) 
                    - \sum_{i=1}^n (0) \qty(\alpha(x_{i-1})-\alpha(x_{i}))\\
                &= \sum_{i=1}^n \alpha(x_{i-1}) - \alpha(x_{i})\\
                &= \alpha(x_n) - \alpha(x_0) = \alpha(b) - \alpha(a) > 0
        \end{align*} Which contradicts the conditions needed for $f$ to be Darboux-Stieltjes integrable with respect to $\alpha$ on $[a,b]$.
    \end{proof}
\end{example}

\end{document}
