% Standard Article Definition
\documentclass[]{article}

% Page Formatting
\usepackage[margin=1in]{geometry}
\setlength\parindent{0pt}

% Graphics
\usepackage{graphicx}

% Math Packages
\usepackage{physics}
\usepackage{amsmath, amsfonts, amssymb, amsthm}
\usepackage{mathtools}

% Extra Packages
\usepackage{listings}
\usepackage{hyperref}

% Section Heading Settings
\usepackage{enumitem}
\renewcommand{\theenumi}{\alph{enumi}}
\renewcommand*{\thesection}{Problem \arabic{section}}
\renewcommand*{\thesubsection}{\alph{subsection})}
\renewcommand*{\thesubsubsection}{}%\quad \quad \roman{subsubsection})}

%Custom Commands
\newcommand{\Rel}{\mathcal{R}}
\newcommand{\R}{\mathbb{R}}
\newcommand{\C}{\mathbb{C}}
\newcommand{\N}{\mathbb{N}}
\newcommand{\Z}{\mathbb{Z}}
\newcommand{\Q}{\mathbb{Q}}

\newcommand{\toI}{\xrightarrow{\textsf{\tiny I}}}
\newcommand{\toS}{\xrightarrow{\textsf{\tiny S}}}
\newcommand{\toB}{\xrightarrow{\textsf{\tiny B}}}

\newcommand{\divisible}{ \ \vdots \ }
\newcommand{\st}{\ : \ }

% Theorem Definition
\newtheorem{definition}{Definition}
\newtheorem{assumption}{Assumption}
\newtheorem{theorem}{Theorem}
\newtheorem{lemma}{Lemma}
\newtheorem{proposition}{Proposition}
\newtheorem{example}{Example}


%opening
\title{MATH 5302 Elementary Analysis II - Homework 5}
\author{Jonas Wagner}
\date{2022, March 25\textsuperscript{th}}

\begin{document}

\maketitle

% Problem 1 ----------------------------------------------
\section{}
Let $f$ be a real-valued bounded function on $[-1,1]$. 
Let\[
    \alpha(x) = \begin{cases}
        0 &\text{if} \ -1 \leq x < 0;\\
        2 &\text{if} \  0 \leq x \leq 1.
    \end{cases}
\]
Assume $f$ is Riemann-Stieljes integrable with respect to $\alpha$ on $[-1,1]$. 
Show that \begin{enumerate}
    \item $f$ is continuous at $0$ from the left.
    \item $\int_{-1}^{1} f(x) \dd{\alpha(x)} = 2 f(0)$.
\end{enumerate}

% Part a
\subsection{$f$ is continuous at $0$ from the left}
\begin{example}
    Let $f : [-1, 1] \to \R$ bounded.
    Let \[
        \alpha(x) = \begin{cases}
            0   & -1 \leq x < 0\\
            2   &  0 \leq x < 1
        \end{cases}
    \] If $f$ is Riemann-Stieljes integrable w.r.t. $\alpha$ on $[-1,1]$, then $f$ is continuous at 0 from the left.
    \begin{proof}

    \end{proof}
\end{example}












% Problem 2 ----------------------------------------------
\newpage
\section{}
Let $f$ and $\alpha$ be real-valued bounded functions on $[a,b]$ and $\alpha$ is increasing. 
Let $L(f,\alpha)$ and $U(f,\alpha)$ represents the lower and upper Darboux-Stieltjes integral of $f$ with respect to $\alpha$ on $[a,b]$, respectively,\begin{enumerate}
    \item Show that $U(f,\alpha) \leq U(\abs{f},\alpha)$.
    \item Is it true that $L(f,\alpha) \leq L(\abs{f},\alpha)$?
\end{enumerate}









% Problem 3 ----------------------------------------------
\newpage
\section{}
Let $\alpha$ be a bounded real-0valued increasing function on $[a,b]$. 
Assume $a < c < b$ and $\alpha$ is continuous at $c$. 
Let \[
    f(x) = \begin{cases}
        1   &\text{if } x = c;\\
        0   &\text{if } x \neq c.
    \end{cases}
\] Show directly that $f$ is Darboux-Stieltjes integrable on $[a,b]$ and $\int_{a}^{b} f(x) \dd{\alpha(x)} = 0$. 
(Do not use Theorem 8.16.)










% Problem 4 ----------------------------------------------
\newpage
\section{}
Let $f$ and $\alpha$ be real-valued bounded functions on $[a,b]$ and $\alpha$ is increasing on $[a,b]$. 
Assume $f$ is Darboux-Stieltjes integrable with respect to $\alpha$ on $[a,b]$. 
Let $[c,d] \subset [a,b]$. 
Show that $f$ is Darboux-Stieltjes integrable with respect to $\alpha$ on $[c,d]$.













% Problem 5 ----------------------------------------------
\newpage
\section{}
Let $\alpha$ be a real-valued bounded function on $[a,b]$ and $\alpha$ is increasing with $\alpha(a) < \alpha(b)$. 
Let \[
    f(x) = \begin{cases}
        1   &\text{if $x$ is rational};\\
        0   &\text{if $x$ is irrational}.\\
    \end{cases}
\] Show that if $\alpha$ is continuous on $[a,b]$, then $f$ is not Darboux-Stieltjes integrable with respect to $\alpha$ on $[a,b]$.










\end{document}
