% Standard Article Definition
\documentclass[]{article}

% Page Formatting
\usepackage[margin=1in]{geometry}
\setlength\parindent{0pt}

% Graphics
\usepackage{graphicx}

% Math Packages
\usepackage{physics}
\usepackage{amsmath, amsfonts, amssymb, amsthm}
\usepackage{mathtools}

% Extra Packages
\usepackage{listings}
\usepackage{hyperref}

% Section Heading Settings
\usepackage{enumitem}
\renewcommand{\theenumi}{\alph{enumi}}
\renewcommand*{\thesection}{Problem \arabic{section}}
\renewcommand*{\thesubsection}{\alph{subsection})}
\renewcommand*{\thesubsubsection}{\quad \quad \roman{subsubsection})}

%Custom Commands
\newcommand{\Rel}{\mathcal{R}}
\newcommand{\R}{\mathbb{R}}
\newcommand{\C}{\mathbb{C}}
\newcommand{\N}{\mathbb{N}}
\newcommand{\Z}{\mathbb{Z}}
\newcommand{\Q}{\mathbb{Q}}

\newcommand{\toI}{\xrightarrow{\textsf{\tiny I}}}
\newcommand{\toS}{\xrightarrow{\textsf{\tiny S}}}
\newcommand{\toB}{\xrightarrow{\textsf{\tiny B}}}

\newcommand{\divisible}{ \ \vdots \ }
\newcommand{\st}{\ : \ }


% Theorem Definition
\newtheorem{definition}{Definition}
\newtheorem{assumption}{Assumption}
\newtheorem{theorem}{Theorem}
\newtheorem{lemma}{Lemma}
\newtheorem{proposition}{Proposition}
\newtheorem{example}{Example}


%opening
\title{MATH 5302 Elementary Analysis II - Homework 2}
\author{Jonas Wagner}
\date{2022, Febuary 8\textsuperscript{th}}

\begin{document}

\maketitle

% Problem 1 ----------------------------------------------
\section{}
Complete the proof of Theorem 2.3 in the lecture notes by showing that a decreasing function on $[a,b]$ is integrable.

\begin{definition}\label{def:increasingAndDecreasing}
    Let $f : \R \to \R$ be a real-valued function.
    \begin{enumerate}
        \item $f$ is \underline{\emph{strictly increasing}} over interval $I$ if\[
            \forall_{x_1, x_2\in I} x_1 < x_2 \implies f(x_1) < f(x_2)  
            \]
        \item $f$ is \underline{\emph{strictly decreasing}} over interval $I$ if\[
            \forall_{x_1, x_2\in I} x_1 < x_2 \implies f(x_1) > f(x_2)  
            \]
        \item $f$ is \underline{\emph{increasing}} over interval $I$ if\[
            \forall_{x_1, x_2\in I} x_1 < x_2 \implies f(x_1) \leq f(x_2)  
            \]
        \item $f$ is \underline{\emph{decreasing}} over interval $I$ if\[
            \forall_{x_1, x_2\in I} x_1 < x_2 \implies f(x_1) \geq f(x_2)  
            \]
    \end{enumerate}
\end{definition}

\begin{definition}
    $f : \R \to \R$ is \underline{\emph{monotone}} over interval $I$ if $f$ is either increasing or decreasing over interval $I$.
\end{definition}

\begin{theorem}
    Theorem 1.4 states:\\
    A bounded function $f : [a,b] \to \R$ is \underline{\emph{integrable}} iff \[
        \forall_{\epsilon>0} 
        \exists_{P = \qty{a=x_0<x_1<\dots<x_n=b}} 
        \st U(f,P) - L(f,P) < \epsilon
    \]
\end{theorem}

\begin{theorem}
    Every monotone function $f$ on $[a,b]$ is integrable.
    \begin{proof}
        \begin{lemma}
            Every increasing function $f$ on $[a,b]$ is integrable.
            \begin{proof}
                Let $f$ be increasing on $[a,b]$.
                Let $\epsilon>0$.
                Let $P = \qty{a=x_0<x_1<\dots<x_n=b}$ be a partition on $[a,b]$ with mesh less than $\cfrac{\epsilon}{f(b)-f(a)}$.
                \begin{align*}
                    U(f,P) - L(f,P) 
                        &= \sum_{i=1}^n M(f,[x_{x-1},x_{i}])(x_{i}-x_{i-1}) 
                            - \sum_{i=1}^n m(f,[x_{x-1},x_{i}])(x_{i}-x_{i-1})\\
                        &= \sum_{i=1}^n f(x_i) (x_{i}-x_{i-1}) 
                            - \sum_{i=1}^n f(x_{i-1}) (x_{i}-x_{i-1})\\
                        &= \sum_{i=1}^n [f(x_i) - f(x_{i-1})] (x_{i}-x_{i-1})\\
                        &< \sum_{i=1}^n -\qty([f(x_{i-1}) - f(x_{i})]) \qty(\cfrac{\epsilon}{f(b)-f(a)})\\
                        &= -\qty(f(b) - f(a)) \qty(\cfrac{\epsilon}{f(b)-f(a)})\\
                        &= \qty(f(a) - f(b)) \qty(\cfrac{\epsilon}{f(b)-f(a)}) = \epsilon
                \end{align*}
            \end{proof}
        \end{lemma}
        \begin{lemma}
            Every decreasing function $f$ on $[a,b]$ is integrable.
            \begin{proof}
                Let $f$ be decreasing on $[a,b]$.
                Let $\epsilon>0$.
                Let $P = \qty{a=x_0<x_1<\dots<x_n=b}$ be a partition on $[a,b]$ with mesh less than $\cfrac{\epsilon}{f(a)-f(b)}$.
                \begin{align*}
                    U(f,P) - L(f,P) 
                        &= \sum_{i=1}^n M(f,[x_{x-1},x_{i}])(x_{i}-x_{i-1}) 
                            - \sum_{i=1}^n m(f,[x_{x-1},x_{i}])(x_{i}-x_{i-1})\\
                        &= \sum_{i=1}^n f(x_i) (x_{i}-x_{i-1}) 
                            - \sum_{i=1}^n f(x_{i-1}) (x_{i}-x_{i-1})\\
                        &= \sum_{i=1}^n [f(x_i) - f(x_{i-1})] (x_{i}-x_{i-1})\\
                        &< \sum_{i=1}^n [f(x_i) - f(x_{i-1})] \qty(\cfrac{\epsilon}{f(b)-f(a)})\\
                        &= \qty(f(a) - f(b)) \qty(\cfrac{\epsilon}{f(a)-f(b)}) = \epsilon
                \end{align*}
            \end{proof}
        \end{lemma}
    \end{proof}
\end{theorem}

% Problem 2 ----------------------------------------------
\newpage
\section{}
Let $f$ be a bounded function on $[a,b]$, so that there exists $B>0$ such that $\abs{f(x)}\leq B$ for all $x\in [a,b]$.

% Problem 2a
\subsection{}
Show\[
    U(f^2,P) - L(f^2,P) \leq 2 B[U(f,P) - L(f,P)]
\] for all partitions $P$ of $[a,b]$.
Hint: $f^2(x) - f^2(y) = (f(x) + f(y)) (f(x)-f(y))$

\begin{theorem}\label{thm:pblm2a}
    \[
        U(f^2,P) - L(f^2,P) \leq 2 B[U(f,P) - L(f,P)]
    \] for all partitions $P=\qty{a=x_0<x_1<\dots<x_n=b}$ of $[a,b]$.
    \begin{proof}
        \begin{align*}
            U(f^2,P) - L(f^2,P) 
                &= \sum_{i=1}^n M(f^2,[x_{x-1},x_{i}])(x_{i}-x_{i-1}) 
                    - \sum_{i=1}^n m(f^2,[x_{x-1},x_{i}])(x_{i}-x_{i-1})\\
                &= \sum_{i=1}^n f^2(x_i) (x_{i}-x_{i-1}) 
                    - \sum_{i=1}^n f^2(x_{i-1}) (x_{i}-x_{i-1})\\
                &= \sum_{i=1}^n [f^2(x_i) - f^2(x_{i-1})] (x_{i}-x_{i-1})\\
                &= \sum_{i=1}^n [(f(x_i) + f(x_{i-1})) (f(x_{i})-f(x_{i-1}))] (x_{i}-x_{i-1})\\
    \intertext{Since $\abs{f(x)}\leq B \forall_{x\in [a,b]}$,}
                &\leq\sum_{i=1}^n [(B + B)) (f(x_{i})-f(x_{i-1}))] (x_{i}-x_{i-1})\\
                &=2B\sum_{i=1}^n [(f(x_{i})-f(x_{i-1}))] (x_{i}-x_{i-1})\\
                &=2B\qty[
                    \sum_{i=1}^n f(x_i) (x_{i}-x_{i-1}) 
                        - \sum_{i=1}^n f(x_{i-1}) (x_{i}-x_{i-1})
                ]\\
                &= 2B\qty[
                    \sum_{i=1}^n M(f,[x_{x-1},x_{i}])(x_{i}-x_{i-1}) 
                        - \sum_{i=1}^n m(f,[x_{x-1},x_{i}])(x_{i}-x_{i-1})
                ]\\
                &= 2 B[U(f,P) - L(f,P)]
        \end{align*}
    \end{proof}
\end{theorem}

\newpage
% Problem 2b
\subsection{}
Show that if $f$ is integrable on $[a,b]$, then $f^2$ is also integrable on $[a,b]$.

\begin{theorem}
    $f$ integrable on $[a,b]$ $\implies$ $f^2$ integrable on $[a,b]$.
    \begin{proof}
        From Theorem \ref{thm:pblm2a}, we have that \[
            U(f^2,P) - L(f^2,P) \leq 2 B [U(f,P) - L(f,P)]
        \] for all partitions $P=\qty{a=x_0<x_1<\dots<x_n=b}$ of $[a,b]$.

        $f$ integrable on $[a,b]$ implies\[
            \exists_{\{P_k\}} \st \lim^{k\to\infty} [U(f,P) - L(f,P)] = 0
        \]

        Therefore,\[
            \exists_{\{P_k\}} \st \lim^{k\to\infty} [U(f^2,P) - L(f^2,P)] \leq 2B [U(f,P) - L(f,P)] = 0
        \]

        Which means that the lower and upper Darboux integrals are equal, 
        $U(f^2) = L(f^2)$ and by definition this means $f^2$ is Darboux integrable.
    \end{proof}
\end{theorem}

% Problem 3 ----------------------------------------------
\newpage
\section{}
Let $f$ be a bounded function on $[a,b]$.
Suppose $f^2$ is integrable on $[a,b]$.
Must $f$ also be integrable on $[a,b]$?

\textbf{Answer:} 
No.

A modification of the rational number indicator function can be shown as a counter example:

Let\[
    f(x) = \begin{cases}
        1 & \text{if} x \in \Q\\
        -1 & \text{if} x \notin \Q
    \end{cases}
\] which is clearly not integrable due to the infinite number of discontinuities.

However, $f^2$ would be defined by \[
    f^2(x) = 1
\] which is clearly integrable.

% Problem 4 ----------------------------------------------
\newpage
\section{}
Suppose that $f$ and $g$ are integrable on $[a,b]$. 
Show that $\max(f,g)$ is also integrable on $[a,b]$.
Hint: Derive and apply the formula:\[
    \max(f,g) = \frac{1}{2} (f+g+\abs{f-g})
\]

\begin{theorem}
    For all functions $f: [a,b] \to \R$ and $g: [a,b] \to \R$ that are integrable on $[a,b]$,
    $\max(f,g)$ is also integrable on $[a,b]$.
    \begin{proof}
        The max function is equal to
        \begin{align*}
            \max(f,g)(x) 
                &= \max(f(x),g(x))\\
                &=\begin{cases}
                    f(x) &f(x) \geq g(x)\\
                    g(x) &g(x) < f(x)
                \end{cases}\\
                &= \begin{cases}
                    g(x) + [f(x) - g(x)] &f(x) \geq g(x)\\
                    f(x) + [g(x) - f(x)] &g(x) < f(x)
                \end{cases}\\
                &= \begin{cases}
                    g(x) + \abs{f(x) - g(x)} &f(x) \geq g(x)\\
                    f(x) + \abs{g(x) - f(x)} &g(x) < f(x)
                \end{cases}\\
                &= \frac{1}{2} \begin{cases}
                    f(x) + g(x) + \abs{f(x) - g(x)} &f(x) \geq g(x)\\
                    f(x) + g(x) + \abs{g(x) - f(x)} &g(x) < f(x)
                \end{cases}\\
                &= \cfrac{f(x) + g(x) + \abs{f(x) - g(x)}}{2}
        \end{align*}
        Since $f$ and $g$ are integrable on $[a,b]$, the following is true:
        \begin{enumerate}
            \item $U(f) = L(f)$
            \item $U(g) = L(g)$
        \end{enumerate}
        
    \end{proof}
\end{theorem}




% Problem 5 ----------------------------------------------
\newpage
\section{}
Suppose $f$ and $g$ are continuous functions on $[a,b]$ such that $\int_{a}^{b} f = \int_{a}^{b} g$. 
Prove there exists $x$ in $(a,b)$ such that $f(x) = g(x)$.










% Consider $f(x) = 2x + 1$ over the interval $[1,3]$.
% Let $P$ be the partition $\{1,1.5,2,3\}$.

% \subsection{}
% \textbf{Problem:}
% Compute $L(f,P), U(f,P),$ and $U(f,P) - L(f,P)$

% \begin{definition}\label{def:darboux_sum}
%     Define bounded function $f : [a,b] \to \R$ and set $S \subseteq [a,b]$.

%     Let $M(f,S) := \sup{f(x) \st x \in S}$ and $m(f,S) = \inf{f(x) \st x \in S}$

%     Define the partition $P$ of $[a,b]$ as\[
%         P = \{a = x_0 < x_1 < \cdots < x_n = b\}
%     \]

%     The \underline{\emph{Upper Darboux Sum}} $U(f,P)$ for $f$ w.r.t. $P$ is defined as\[
%         U(f,P) = \sum_{i=1}^{n} M(f,[x_{i-1},x_i]) \cdot (x_i - x_{i-1})
%     \]

%     The \underline{\emph{Lower Darboux Sum}} $L(f,P)$ for $f$ w.r.t. $P$ is defined as\[
%         L(f,P) = \sum_{i=1}^{n} m(f,[x_{i-1},x_i]) \cdot (x_i - x_{i-1})
%     \]
% \end{definition}

% \textbf{Solution:}
% Let $f: [1, 3] \to \R$ defined by \[
%     f(x) = 2x + 1
% \] and partition $P$ of $[1,3]$ defined as \[
%     \{1,1.5,2,3\}
% \]

% \begin{align*}
%     L(f,P) &= \sum_{i=1}^{n} m(f,[x_{i-1},x_i]) \cdot (x_i - x_{i-1})\\
%         &= \sum_{i=1}^{3} m(2 x + 1,[x_{i-1},x_i]) \cdot (x_i - x_{i-1})\\
%         &= m(2x + 1, [1, 1.5]) * (1.5 - 1)
%             + m(2x + 1, [1.5, 2]) * (2 - 1.5)
%             + m(2x + 1, [2, 3]) * (3 - 2)\\
%         &= 3 (0.5) 
%             + 4 (0.5)
%             + 5 (1)\\
%     \Aboxed{L(f,P) &= 8.5}
% \end{align*}

% \begin{align*}
%     U(f,P) &= \sum_{i=1}^{n} M(f,[x_{i-1},x_i]) \cdot (x_i - x_{i-1})\\
%         &= \sum_{i=1}^{3} M(2 x + 1,[x_{i-1},x_i]) \cdot (x_i - x_{i-1})\\
%         &= M(2x + 1, [1, 1.5]) * (1.5 - 1)
%             + M(2x + 1, [1.5, 2]) * (2 - 1.5)
%             + M(2x + 1, [2, 3]) * (3 - 2)\\
%         &= 4 (0.5) 
%             + 5 (0.5)
%             + 7 (1)\\
%     \Aboxed{L(f,P) &= 11.5}
% \end{align*}

% \begin{align*}
%     \Aboxed{U(f,p) - L(f,p) &= 11.5 - 8.5 = 3}
% \end{align*}

% \subsection{}
% \textbf{Problem:}
% What happens to the value of $U(f,P) - L(f,P)$?

% \textbf{Solution:} it gets smaller

% \textbf{Proof:} 
% \begin{align*}
%     L(f,P) &= \sum_{i=1}^{n} m(f,[x_{i-1},x_i]) \cdot (x_i - x_{i-1})\\
%         &= \sum_{i=1}^{4} m(2 x + 1,[x_{i-1},x_i]) \cdot (x_i - x_{i-1})\\
%         &= m(2x + 1, [1, 1.5]) * (1.5 - 1)
%             + m(2x + 1, [1.5, 2]) * (2 - 1.5)\\
%             &\quad+ m(2x + 1, [2, 2.5]) * (2.5 - 2)
%                 + m(2x + 1, [2.5, 3]) * (3 - 2.5)\\
%         &= 3 (0.5) 
%             + 4 (0.5)
%             + 5 (0.5)
%             + 6 (0.5)\\
%     \Aboxed{L(f,P) &= 9}
% \end{align*}

% \begin{align*}
%     U(f,P) &= \sum_{i=1}^{n} M(f,[x_{i-1},x_i]) \cdot (x_i - x_{i-1})\\
%         &= \sum_{i=1}^{3} M(2 x + 1,[x_{i-1},x_i]) \cdot (x_i - x_{i-1})\\
%         &= M(2x + 1, [1, 1.5]) * (1.5 - 1)
%             + M(2x + 1, [1.5, 2]) * (2 - 1.5)\\
%             &\quad+ M(2x + 1, [2, 2.5]) * (2.5 - 2)
%                 + M(2x + 1, [2.5, 3]) * (3 - 2.5)\\
%         &= 4 (0.5) 
%             + 5 (0.5)
%             + 6 (0.5)
%             + 7 (0.5)\\
%     \Aboxed{L(f,P) &= 11}
% \end{align*}

% \begin{align*}
%     \Aboxed{U(f,p) - L(f,p) &= 11 - 9 = 2}
% \end{align*}

% \newpage
% \subsection{}
% \textbf{Problem:}
% Find a partition $P'$ of $[1,3]$ for which $U(L,P') - L(f,P') < 2$

% \textbf{Solution:}
% Let \[
%     P' = \qty{1, 1.4, 1.8, 2.2, 2.6, 3}
% \]

% \begin{align*}
%     L(f,P') &= \sum_{i=1}^{n} m(f,[x_{i-1},x_i]) \cdot (x_i - x_{i-1})\\
%         &= \sum_{i=1}^{5} m(2 x + 1,[x_{i-1},x_i]) \cdot (x_i - x_{i-1})\\
%         &= m(2x + 1, [1, 1.4]) * (1.4 - 1)\\
%             &\quad  + m(2x  + 1, [1.4, 1.8]) * (1.8 - 1.4)\\
%             &\quad  + m(2x + 1, [1.8, 2.2]) * (2.2 - 1.8)\\
%             &\quad  + m(2x + 1, [2.2, 2.6]) * (2.6 - 2.2)\\
%             &\quad  + m(2x + 1, [2.6, 3]) * (3 - 2.6)\\
%         &= 3 (0.4) 
%             + 3.8 (0.4)
%             + 4.6 (0.4)
%             + 5.4 (0.4)
%             + 6.2 (0.4)\\
%     \Aboxed{L(f,P') &= 9.2}
% \end{align*}

% \begin{align*}
%     U(f,P') &= \sum_{i=1}^{n} M(f,[x_{i-1},x_i]) \cdot (x_i - x_{i-1})\\
%         &= \sum_{i=1}^{5} M(2 x + 1,[x_{i-1},x_i]) \cdot (x_i - x_{i-1})\\
%         &= M(2x + 1, [1, 1.4]) * (1.4 - 1)\\
%             &\quad  + M(2x  + 1, [1.4, 1.8]) * (1.8 - 1.4)\\
%             &\quad  + M(2x + 1, [1.8, 2.2]) * (2.2 - 1.8)\\
%             &\quad  + M(2x + 1, [2.2, 2.6]) * (2.6 - 2.2)\\
%             &\quad  + M(2x + 1, [2.6, 3]) * (3 - 2.6)\\
%         &= 3.8 (0.4) 
%             + 4.6 (0.4)
%             + 5.4 (0.4)
%             + 6.2 (0.4)
%             + 7 (0.4)\\
%     \Aboxed{U(f,P') &= 10.8}
% \end{align*}

% \begin{align*}
%     \Aboxed{U(f,p) - L(f,p) &= 10.8 - 9.2 = 1.6}
% \end{align*}



% %%%%%%%%%%%%%%%%%%%%%%%%%%%%
% % Future reader... I spent too long on this...
% %%%%%%%%%%%%%%%%%%%%%%%%%%%%%%%%%%%%%%%
% % Define \[
% %     P_n = \qty{\frac{2i}{n} + 1 \st i = 0, \dots, n}
% % \]

% % \begin{align*}
% %     U(f,P_n) &= \sum_{i=1}^{n} M(f,[x_{i-1},x_i]) \cdot (x_i - x_{i-1})\\
% %         &= \sum_{i=1}^n M(2x + 1, [\frac{2i-2}{n} + 1, \frac{2i}{n} + 1]) 
% %             \cdot (\frac{2i}{n} + 1 - (\frac{2i-2}{n} + 1))\\
% % \intertext{since $f(x)$ is a monotopically increasing function, 
% %             $M(f,[x_{i-1},x_i]) = f(x_i)$}
% %         &= \sum_{i=1}^n (2*(\frac{2i}{n} + 1) + 2)* \frac{2}{n}\\
% %         &= \frac{2}{n} \sum_{i=1}^n \frac{4 i}{n} + 4\\
% %         &= \frac{8}{n} + \frac{4}{n^2} \sum_{i=1}^n i\\
% %         &= \frac{8}{n} + \frac{4}{n^2} \frac{n (n+1)}{2}\\
% %         &= \frac{8}{n} + \frac{4n (n+1)}{2n^2}\\
% %         &= \frac{8}{n} + \frac{2(n+1)}{n}\\
% %         &= \frac{8}{n} + \frac{2}{n} + 2\\
% %         &= 2 + \frac{8}{n}
% % \end{align*}

% % \begin{align*}
% %     L(f,P_n) &= \sum_{i=1}^{n} m(f,[x_{i-1},x_i]) \cdot (x_i - x_{i-1})\\
% %         &= \sum_{i=1}^n m(2x + 1, [\frac{i-1}{n} + 1, \frac{i}{n} + 1]) 
% %             \cdot (\frac{i}{n} + 1 - (\frac{i-1}{n} + 1))\\
% % \intertext{since $f(x)$ is a monotopically increasing function, 
% %             $M(f,[x_{i-1},x_i]) = f(x_{i-1})$}
% %         &= \sum_{i=1}^n (2*(\frac{i-1}{n} + 1) + 1)* \frac{1}{n}\\
% %         &= \frac{1}{n} \sum_{i=1}^n \frac{2 i - 2}{n} + 3\\
% %         &= \frac{3}{n} + \frac{2}{n^2} \sum_{i=1}^n (i - 1)\\
% %         &= \frac{3}{n} - \frac{2n}{n^2} + \frac{2}{n^2} \sum_{i=1}^n i\\
% %         &= \frac{3}{n} - \frac{2}{n} + \frac{2n(n+1)}{2 n^2}\\
% %         &= \frac{1}{n} + \frac{n+1}{n}\\
% %         &= 1 + \frac{2}{n}
% % \end{align*}

% % \begin{align*}
% %     % \Aboxed{
% %         U(f,P_n) - L(f,P_n) 
% %             &= \qty(1 + \frac{4}{n}) - \qty(1 + \frac{2}{n})\\
% %             &= \frac{2}{n}
% %     % }
% % \end{align*}

% % Therefore, 




% % Problem 2
% \newpage
% \section{}
% Let \[
%     f(x) = \begin{cases}
%         x & \text{if $x$ is rational on $[0,1]$}\\
%         0 & \text{if $x$ is irrational on $[0,1]$}
%     \end{cases}
% \]

% \subsection{}
% \textbf{Problem:}
% Find the upper and lower Darboux integrals for $f$ on the interval $[0,1]$.

% \begin{definition}\label{def:darboux_int}
%     Define bounded function $f : [a,b] \to \R$ and set $S \subseteq [a,b]$.

%     Let $M(f,S) := \sup{f(x) \st x \in S}$ and $m(f,S) = \inf{f(x) \st x \in S}$

%     Let $U(f,P)$ and $L(f,P)$ for $f$ w.r.t. $P$ be defined by\[
%         U(f,P) = \sum_{i=1}^{n} M(f,[x_{i-1},x_i]) \cdot (x_i - x_{i-1})
%     \] and \[
%         L(f,P) = \sum_{i=1}^{n} m(f,[x_{i-1},x_i]) \cdot (x_i - x_{i-1})
%     \]

%     The \underline{\emph{Upper Darboux Integral}} $U(f)$ for $f$ over $[a,b]$ is defined as \[
%         U(f) = \inf\qty{U(f,P) \st P = \{a = x_0 < x_1 < \cdots < x_n = b\}}
%     \]

%     The \underline{\emph{Lower Darboux Integral}} $L(f)$ for $f$ over $[a,b]$ is defined as \[
%         L(f) = \sup\qty{L(f,P) \st P = \{a = x_0 < x_1 < \cdots < x_n = b\}}
%     \]

% \end{definition}

% \textbf{Solution:}
% Let $P_1 = {0,1}$,
% \begin{align*}
%     U(f,P_1) &= \sum_{i=1}^{n} M(f,[x_{i-1},x_i]) \cdot (x_i - x_{i-1})\\
%         &= M(f,[0,1]) * (1 - 0)\\
%         &= 1 * 1\\
%     U(f,P_1) &= 1
% \end{align*}
% \begin{align*}
%     L(f,P_1) &= \sum_{i=1}^{n} m(f,[x_{i-1},x_i]) \cdot (x_i - x_{i-1})\\
%         &= m(f,[0,1]) * (1 - 0)\\
%         &= 0 * 1\\
%     L(f,P_1) &= 0
% \end{align*}

% $U(f)$ is bounded by $U(f,P_1)$,\[
%     U(f) = \inf_{P}\qty{U(f,P)} \leq U(f, P_1) = 1
% \]
% Similarly, $L(f)$ is bounded by $L(f,P_1)$,\[
%     L(f) = \sup_{P}\qty{L(f,P)} \geq L(f, P_1) = 0
% \]

% Let $P_n = \qty{\frac{i}{n}, \forall_{i = 0,\dots,n}}$,
% \begin{align*}
%     U(f,P_n) &= \sum_{i=1}^{n} M(f,[x_{i-1},x_i]) \cdot (x_i - x_{i-1})\\
%         &= \sum_{i=1}^{n} M(f, [\frac{i-1}{n}, \frac{i}{n}]) * (\frac{i}{n} - \frac{i-1}{n})\\
%         &= \sum_{i=1}^{n} \frac{i}{n} * (\frac{1}{n})\\
%         &= \frac{1}{n^2} \sum_{i=1}^{n} i\\
%         &= \frac{1}{n^2} \frac{n(n+1)}{2}\\
%         &= \frac{n+1}{2n}\\
%     U(f,P_n) &= \frac{1}{2} + \frac{1}{2n}
% \end{align*}
% \begin{align*}
%     L(f,P_n) &= \sum_{i=1}^{n} m(f,[x_{i-1},x_i]) \cdot (x_i - x_{i-1})\\
%         &= \sum_{i=1}^{n} m(f, [\frac{i-1}{n}, \frac{i}{n}]) * (\frac{i}{n} - \frac{i-1}{n})\\
%         &= \sum_{i=1}^{n} 0 * (\frac{1}{n})\\
%     L(f,P_n) &= 0
% \end{align*}

% $U(f)$ is bounded by $U(f,P_n)$,\[
%     U(f) = \inf_{P} \qty{U(f,P)} \leq U(f,P_n) = \frac{1}{2} + \frac{1}{2n}
% \] which when taking $n$ to the limit results in \[
%     \boxed{U(f) \leq \lim_{n\to\infty} \frac{1}{2} + \frac{1}{2n} = \frac{1}{2}}
% \]

% $L(f)$ is bounded by $L(f,P_n)$,\[
%     L(f) = \sup_{P} \qty{L(f,P)} \geq L(f,P_n) = 0
% \]
% For $f$, the definitions of $L(f,P)$ and $m(f, [a,b])$ actually demonstrate that $L(f,P) = 0 \forall_{P}$. 
% The definition of $L(f)$ then implies\[
%     \boxed{L(f) = \sup_{P} \qty(L(f,P) = 0) = 0}
% \]

% \subsection{}
% \textbf{Problem:}
% Is $f$ integrable on $[0,1]$?

% \begin{definition}\label{def:darboux_integrable}
%     $f$ is \emph{\underline{Darboux Integrable}} on $[a,b]$ iff $L(f) = U(f)$.
%     i.e.\[
%         \int_{a}^{b} f = \int_{a}^{b} f(x) \dd x = L(f) = U(f)
%     \]
% \end{definition}

% \textbf{Answer:}
% No

% \textbf{Proof:}
% \[
%     L(f) = 0 \neq \frac{1}{2} = U(f)
% \]

% % Problem 3
% \newpage
% \section{}
% Let \[
%     f(x) = \begin{cases}
%         1 & \text{if $\exists_{n \in \N} \st x = \frac{1}{n}$}\\
%         0 & \text{otherwise}
%     \end{cases}
% \]
% Show that $f$ is integrable on $[0,1]$ and compute $\int_{0}^{1} f$.

% \textbf{Solution:}
% Let $P_n = \qty{\frac{i}{n}, \forall_{i = 0,\dots,n}}$,
% \begin{align*}
%     U(f,P_n) &= \sum_{i=1}^{n} M(f,[x_{i-1},x_i]) \cdot (x_i - x_{i-1})\\
%         &= \sum_{i=1}^{n} M(f, [\frac{i-1}{n}, \frac{i}{n}]) * (\frac{i}{n} - \frac{i-1}{n})\\
%         &= \frac{1}{n}\sum_{i=1}^{n} M(f, [\frac{i-1}{n}, \frac{i}{n}])\\
%     \intertext{
%         $M(f, [a,b])$ is 0 unless $\qty{\frac{1}{n} \st \exists_{n \in \N}} \cap [a,b] \neq \emptyset$
%     }
%         &= \frac{1}{n}\sum_{i=1}^{n} \begin{cases}
%                 1 & \exists_{k \in \N} \st \frac{1}{k} \in [\frac{i-1}{n}, \frac{i}{n}]\\
%                 0 & \text{otherwise}
%             \end{cases}
% \end{align*}

% As $n$ increases, fewer portions of the partition have $M(f,[x_{i-1},x_{i}]) = 1$.
% Since as $n$ increases, the size of each partition also gets smaller,\[
%     \lim_{n\to\infty} U(f,P_n) = 0
% \]
% Therefore, by definition, \[
%     \boxed{U(f) = \inf_{P} \qty{U(f,P)} = 0}
% \]

% \begin{align*}
%     L(f,P_n) &= \sum_{i=1}^{n} m(f,[x_{i-1},x_i]) \cdot (x_i - x_{i-1})\\
%         &= \sum_{i=1}^{n} m(f, [\frac{i-1}{n}, \frac{i}{n}]) * (\frac{i}{n} - \frac{i-1}{n})\\
%         &= \sum_{i=1}^{n} 0 * (\frac{1}{n})\\
%     L(f,P_n) &= 0
% \end{align*}
% $L(f)$ is bounded by $L(f,P_n)$,\[
%     L(f) = \sup_{P} \qty{L(f,P)} \geq L(f,P_n) = 0
% \]
% % For $f$, the definitions of $L(f,P)$ and $m(f, [a,b])$ actually demonstrate that $L(f,P) = 0 \forall_{P}$. 
% The definition of $L(f)$ then implies\[
%     \boxed{L(f) = \sup_{P} \qty(L(f,P) = 0) = 0}
% \]

% \textbf{Answer:}
% $f$ is integrable on $[0,1]$ since $L(f) = U(f)$.

% The integral is given as: \[
%     \int_{0}^{1} f = L(f) = U(f) = 0
% \]

% % Problem 4
% \newpage
% \section{}
% \textbf{Problem:}
% Let $f$ be a bounded function on $[a,b]$. 
% Supposes there exist sequences $(U_n)$ and $(L_n)$ of upper and lower Darboux sums for $f$ such that $\lim_{n\to\infty} (U_n - L_n) = 0$.
% Show that $f$ is integrable and $\int_{a}^{b} f = \lim_{n\to\infty} U_n = \lim_{n\to\infty} L_n$.

% % \textbf{Solution:}
% % Let $f : [a,b] \to \R$ be bounded $(\exists_{M \in \R} \st \forall_{x \in [a,b]} f(x) < M)$. 

% % Let sequences $(U_n)$ and $(L_n)$ be defined as, \footnote{
% %     Making the assumption that $U_n$ and $L_n$ share the same partition for each element $n$, 
% %     which may restrict the generality, but likely does not effect the proof
% % } \[
% %     \{U_n \st U_n = U(f,P_n)\}
% % \] and \[
% %     \{L_n \st L_n = L(f,P_n)\}
% % \] so that \[
% %     \lim_{n\to\infty} (U_n - L_n) = 0
% % \]

% \begin{theorem}
%     For bounded function $f : [a,b] \to \R$, if \[
%         \exists_{(U_n),(L_n)} \st \lim_{n\to\infty} (U_n - L_n) = 0
%     \] then $f$ is integrable and \[
%         \int_{a}^{b} f = \lim_{n\to\infty} U_n = \lim_{n\to\infty} L_n
%     \]
%     \begin{proof}
%         By Definition \ref{def:darboux_integrable}, 
%         $f$ is integrable iff $L(f) = U(f)$.

%         Fundamentally this is proven by the squeeze theorem.
%         From Definition \ref{def:darboux_int}, \[
%             L(f,P_l) \leq L(f) = \int_{a}^{b} f = U(f) \leq U(f,P_u)
%         \] for all partitions of $[a,b]$, $P_l$ and $P_u$.

%         The convergence of Darboux sum sequences $U_n$ and $L_n$ to the same point, \[
%             \lim_{n\to\infty} (U_n - L_n) = 0
%         \] "squeezes" the Darboux integrals to the same point.

%         \begin{gather*}
%             L_n = L(f,P_n) \leq L(f) = \int_{a}^{b} f = U(f) \leq U(f,P_n) = U_n\\
%             \lim_{n\to\infty} \qty{
%                 L_n \leq L(f) = U(f) \leq U_n
%             }
%         \intertext{Since $\lim_{n\to\infty} L_n = \lim_{n\to\infty} U_n$,}
%             \boxed{\lim_{n\to\infty} L_n = L(f) = \int_{a}^{b} f = U(f) = \lim_{n\to\infty}}
%         \end{gather*}
        
%     \end{proof}
% \end{theorem}


% % Problem 5
% \newpage
% \section{}
% Let $f$ be integrable on $[a,b]$, and suppose $g$ is a function on $[a,b]$ such that $g(x) = f(x)$ except for finitely many $x$ in $[a,b]$.
% Show that $g$ is integrable and $\int_{a}^{b} f = \int_{a}^{b} g$.

% \begin{theorem}
%     For integrable function $f : [a,b] \to \R$ and $g : [a,b] \to \R$,\[
%         f(x) = g(x) \forall_{x \in [a,b] \backslash S}
%     \] where $S$ is finite, then 
%     \begin{enumerate}
%         \item $g$ is Darboux integrable
%         \item $\int_{a}^{b} f = \int_{a}^{b} g$
%     \end{enumerate} 
%     \begin{proof}
%         By Definition \ref{def:darboux_integrable}, $f$ is integrable iff $L(f) = U(f)$.
%         In addition, Definition \ref{def:darboux_int} implies \[
%             L(f,P_l) \leq L(f) = \int_{a}^{b} f = U(f) \leq U(f,P_u)
%         \] for all partitions of $[a,b]$, $P_l$ and $P_u$.

%         From Definition \ref{def:darboux_sum}, for partition $P = \{a = x_0 < \cdots < x_n = b\}$, \[
%             L(f,P) = \sum_{i=1}^{n} m(f, [x_{i-1},x_{i}]) \cdot (x_{i} - x_{i-1})
%         \] and \[
%             U(f,P) = \sum_{i=1}^{n} M(f, [x_{i-1},x_{i}]) \cdot (x_{i} - x_{i-1})
%         \]

%         When comparing the $L(f,P)$ to $L(g,p)$ or $U(f,P)$ to $U(g,P)$,
%         we can decompose the partition segment with any differences between $m(\cdot)$ and $M(\cdot)$.
%         \begin{align*}
%             L(g,P_n) &= \sum_{i=1}^{n} m(g, [x_{i-1},x_{i}]) \cdot (x_{i} - x_{i-1})\\
%                 &= \sum_{i=1}^{n} \qty(m(f, [x_{i-1},x_{i}]) + \Delta_{i}) \cdot (x_{i} - x_{i-1})\\
%                 &= L(f,P) + \sum_{i=1}^{n} \Delta_{i} \cdot (x_{i} - x_{i-1})\\
%         \intertext{Since $S$ is a finite set of points in which $f(x) \neq g(x)$,}
%             \lim_{n\to\infty} L(g,P_n) &= \lim_{n\to\infty} L(f,P_n) + \lim_{n\to\infty} \sum_{i=1}^{n} \Delta_{i} \cdot (x_{i} - x_{i-1})\\
%             \Aboxed{L(g) = \lim_{n\to\infty} L(g,P_n) &= \lim_{n\to\infty} L(f,P_n) + 0 = L(f)}
%         \end{align*}
%         \begin{align*}
%             LU(g,P_n) &= \sum_{i=1}^{n} M(g, [x_{i-1},x_{i}]) \cdot (x_{i} - x_{i-1})\\
%                 &= \sum_{i=1}^{n} \qty(M(f, [x_{i-1},x_{i}]) + \Delta_{i}) \cdot (x_{i} - x_{i-1})\\
%                 &= U(f,P) + \sum_{i=1}^{n} \Delta_{i} \cdot (x_{i} - x_{i-1})\\
%         \intertext{Since $S$ is a finite set of points in which $f(x) \neq g(x)$,}
%             \lim_{n\to\infty} U(g,P_n) &= \lim_{n\to\infty} U(f,P_n) + \lim_{n\to\infty} \sum_{i=1}^{n} \Delta_{i} \cdot (x_{i} - x_{i-1})\\
%             \Aboxed{U(g) = \lim_{n\to\infty} U(g,P_n) &= \lim_{n\to\infty} U(f,P_n) + 0 = U(f)}
%         \end{align*}

%         Since $L(g) = L(f)$, $U(g) = U(f)$, and $L(f) = \int_{a}^{b} f = U(f)$,\[
%             L(g) = L(f) = \int_{a}^{b} f = U(f) = U(g)
%         \]
%         These equivalences mean that $g$ is Darboux integrable since \[
%             \boxed{L(g) = \int_{a}^{b} g = U(g)}
%         \] and that \[
%             \boxed{\int_{a}^{b} f = \int_{a}^{b} g}
%         \]
%     \end{proof}
% \end{theorem}














\end{document}
