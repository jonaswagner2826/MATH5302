% Standard Article Definition
\documentclass[]{article}

% Page Formatting
\usepackage[margin=1in]{geometry}
\setlength\parindent{0pt}

% Graphics
\usepackage{graphicx}

% Math Packages
\usepackage{physics}
\usepackage{amsmath, amsfonts, amssymb, amsthm}
\usepackage{mathtools}

% Extra Packages
\usepackage{listings}
\usepackage{hyperref}

% Section Heading Settings
\usepackage{enumitem}
\renewcommand{\theenumi}{\alph{enumi}}
\renewcommand*{\thesection}{Problem \arabic{section}}
\renewcommand*{\thesubsection}{\alph{subsection})}
\renewcommand*{\thesubsubsection}{\quad \quad \roman{subsubsection})}

%Custom Commands
\newcommand{\Rel}{\mathcal{R}}
\newcommand{\R}{\mathbb{R}}
\newcommand{\C}{\mathbb{C}}
\newcommand{\N}{\mathbb{N}}
\newcommand{\Z}{\mathbb{Z}}
\newcommand{\Q}{\mathbb{Q}}

\newcommand{\toI}{\xrightarrow{\textsf{\tiny I}}}
\newcommand{\toS}{\xrightarrow{\textsf{\tiny S}}}
\newcommand{\toB}{\xrightarrow{\textsf{\tiny B}}}

\newcommand{\divisible}{ \ \vdots \ }
\newcommand{\st}{\ : \ }


% Theorem Definition
\newtheorem{definition}{Definition}
\newtheorem{assumption}{Assumption}
\newtheorem{theorem}{Theorem}
\newtheorem{lemma}{Lemma}
\newtheorem{proposition}{Proposition}
\newtheorem{example}{Example}


%opening
\title{MATH 5302 Elementary Analysis II - Homework 2}
\author{Jonas Wagner}
\date{2022, Febuary 9\textsuperscript{th}}

\begin{document}

\maketitle

% Problem 1 ----------------------------------------------
\section{}
Complete the proof of Theorem 2.3 in the lecture notes by showing that a decreasing function on $[a,b]$ is integrable.

\begin{definition}\label{def:increasingAndDecreasing}
    Let $f : \R \to \R$ be a real-valued function.
    \begin{enumerate}
        \item $f$ is \underline{\emph{strictly increasing}} over interval $I$ if\[
            \forall_{x_1, x_2\in I} x_1 < x_2 \implies f(x_1) < f(x_2)  
            \]
        \item $f$ is \underline{\emph{strictly decreasing}} over interval $I$ if\[
            \forall_{x_1, x_2\in I} x_1 < x_2 \implies f(x_1) > f(x_2)  
            \]
        \item $f$ is \underline{\emph{increasing}} over interval $I$ if\[
            \forall_{x_1, x_2\in I} x_1 < x_2 \implies f(x_1) \leq f(x_2)  
            \]
        \item $f$ is \underline{\emph{decreasing}} over interval $I$ if\[
            \forall_{x_1, x_2\in I} x_1 < x_2 \implies f(x_1) \geq f(x_2)  
            \]
    \end{enumerate}
\end{definition}

\begin{definition}
    $f : \R \to \R$ is \underline{\emph{monotone}} over interval $I$ if $f$ is either increasing or decreasing over interval $I$.
\end{definition}

\begin{theorem}
    Theorem 1.4 states:\\
    A bounded function $f : [a,b] \to \R$ is \underline{\emph{integrable}} iff \[
        \forall_{\epsilon>0} 
        \exists_{P = \qty{a=x_0<x_1<\dots<x_n=b}} 
        \st U(f,P) - L(f,P) < \epsilon
    \]
\end{theorem}

\begin{theorem}
    Every monotone function $f$ on $[a,b]$ is integrable.
    \begin{proof}
        \begin{lemma}
            Every increasing function $f$ on $[a,b]$ is integrable.
            \begin{proof}
                Proved in class
            \end{proof}
        \end{lemma}
        \begin{lemma}
            Every decreasing function $f$ on $[a,b]$ is integrable.
            \begin{proof}
                Let $f$ be decreasing on $[a,b]$.
                Let $\epsilon>0$.
                Let $P = \qty{a=x_0<x_1<\dots<x_n=b}$ be a partition on $[a,b]$ with mesh less than $\cfrac{\epsilon}{f(a)-f(b)}$.
                \begin{align*}
                    U(f,P) - L(f,P) 
                        &= \sum_{i=1}^n M(f,[x_{x-1},x_{i}])(x_{i}-x_{i-1}) 
                            - \sum_{i=1}^n m(f,[x_{x-1},x_{i}])(x_{i}-x_{i-1})\\
                        &= \sum_{i=1}^n f(x_i) (x_{i}-x_{i-1}) 
                            - \sum_{i=1}^n f(x_{i-1}) (x_{i}-x_{i-1})\\
                        &= \sum_{i=1}^n [f(x_i) - f(x_{i-1})] (x_{i}-x_{i-1})\\
                        &< \sum_{i=1}^n [f(x_i) - f(x_{i-1})] \qty(\cfrac{\epsilon}{f(a)-f(b)})\\
                        &= \sum_{i=1}^n -[f(x_{i-1}) - f(x_{i})] \qty(\cfrac{\epsilon}{f(a)-f(b)})\\
                        &= \qty(f(a) - f(b)) \qty(\cfrac{\epsilon}{f(a)-f(b)}) = \epsilon
                \end{align*}
            \end{proof}
        \end{lemma}
    \end{proof}
\end{theorem}

% Problem 2 ----------------------------------------------
\newpage
\section{}
Let $f$ be a bounded function on $[a,b]$, so that there exists $B>0$ such that $\abs{f(x)}\leq B$ for all $x\in [a,b]$.

% Problem 2a
\subsection{}
Show\[
    U(f^2,P) - L(f^2,P) \leq 2 B[U(f,P) - L(f,P)]
\] for all partitions $P$ of $[a,b]$.
Hint: $f^2(x) - f^2(y) = (f(x) + f(y)) (f(x)-f(y))$

\begin{theorem}\label{thm:pblm2a}
    \[
        U(f^2,P) - L(f^2,P) \leq 2 B[U(f,P) - L(f,P)]
    \] for all partitions $P=\qty{a=x_0<x_1<\dots<x_n=b}$ of $[a,b]$.
    \begin{proof}
        \begin{align*}
            U(f^2,P) - L(f^2,P) 
                &= \sum_{i=1}^n M(f^2,[x_{x-1},x_{i}])(x_{i}-x_{i-1}) 
                    - \sum_{i=1}^n m(f^2,[x_{x-1},x_{i}])(x_{i}-x_{i-1})\\
                &= \sum_{i=1}^n f^2(x_i) (x_{i}-x_{i-1}) 
                    - \sum_{i=1}^n f^2(x_{i-1}) (x_{i}-x_{i-1})\\
                &= \sum_{i=1}^n [f^2(x_i) - f^2(x_{i-1})] (x_{i}-x_{i-1})\\
                &= \sum_{i=1}^n [(f(x_i) + f(x_{i-1})) (f(x_{i})-f(x_{i-1}))] (x_{i}-x_{i-1})\\
    \intertext{Since $\abs{f(x)}\leq B \forall_{x\in [a,b]}$,}
                &\leq\sum_{i=1}^n [(B + B)) (f(x_{i})-f(x_{i-1}))] (x_{i}-x_{i-1})\\
                &=2B\sum_{i=1}^n [(f(x_{i})-f(x_{i-1}))] (x_{i}-x_{i-1})\\
                &=2B\qty[
                    \sum_{i=1}^n f(x_i) (x_{i}-x_{i-1}) 
                        - \sum_{i=1}^n f(x_{i-1}) (x_{i}-x_{i-1})
                ]\\
                &= 2B\qty[
                    \sum_{i=1}^n M(f,[x_{x-1},x_{i}])(x_{i}-x_{i-1}) 
                        - \sum_{i=1}^n m(f,[x_{x-1},x_{i}])(x_{i}-x_{i-1})
                ]\\
                &= 2 B[U(f,P) - L(f,P)]
        \end{align*}
    \end{proof}
\end{theorem}

\newpage
% Problem 2b
\subsection{}
Show that if $f$ is integrable on $[a,b]$, then $f^2$ is also integrable on $[a,b]$.

\begin{theorem}
    $f$ integrable on $[a,b]$ $\implies$ $f^2$ integrable on $[a,b]$.
    \begin{proof}
        From Theorem \ref{thm:pblm2a}, we have that \[
            U(f^2,P) - L(f^2,P) \leq 2 B [U(f,P) - L(f,P)]
        \] for all partitions $P=\qty{a=x_0<x_1<\dots<x_n=b}$ of $[a,b]$.

        $f$ integrable on $[a,b]$ implies\[
            \exists_{\{P_k\}} \st \lim^{k\to\infty} [U(f,P) - L(f,P)] = 0
        \]

        Therefore,\[
            \exists_{\{P_k\}} \st \lim^{k\to\infty} [U(f^2,P) - L(f^2,P)] \leq 2B [U(f,P) - L(f,P)] = 0
        \]

        Which means that the lower and upper Darboux integrals are equal, 
        $U(f^2) = L(f^2)$ and by definition this means $f^2$ is Darboux integrable.
    \end{proof}
\end{theorem}

% Problem 3 ----------------------------------------------
\newpage
\section{}
Let $f$ be a bounded function on $[a,b]$.
Suppose $f^2$ is integrable on $[a,b]$.
Must $f$ also be integrable on $[a,b]$?

\textbf{Answer:} 
No.

A modification of the rational number indicator function can be shown as a counter example:

Let\[
    f(x) = \begin{cases}
        1 & \text{if} x \in \Q\\
        -1 & \text{if} x \notin \Q
    \end{cases}
\] which is clearly not integrable due to the infinite number of discontinuities.

However, $f^2$ would be defined by \[
    f^2(x) = 1
\] which is clearly integrable.

% Problem 4 ----------------------------------------------
\newpage
\section{}
Suppose that $f$ and $g$ are integrable on $[a,b]$. 
Show that $\max(f,g)$ is also integrable on $[a,b]$.
Hint: Derive and apply the formula:\[
    \max(f,g) = \frac{1}{2} (f+g+\abs{f-g})
\]

\begin{theorem}
    For all functions $f: [a,b] \to \R$ and $g: [a,b] \to \R$ that are integrable on $[a,b]$,
    $\max(f,g)$ is also integrable on $[a,b]$.
    \begin{proof}
        Let function $h$ be defined as\[
            h(x) := \max(f(x),g(x))
        \] which is the same as $\max(f,g)$.
        \begin{align*}
            \max(f,g)(x) 
                &= h(x) = \max(f(x),g(x))\\
                &=\begin{cases}
                    f(x) &f(x) \geq g(x)\\
                    g(x) &g(x) < f(x)
                \end{cases}\\
                &= \begin{cases}
                    g(x) + [f(x) - g(x)] &f(x) \geq g(x)\\
                    f(x) + [g(x) - f(x)] &g(x) < f(x)
                \end{cases}\\
                &= \begin{cases}
                    g(x) + \abs{f(x) - g(x)} &f(x) \geq g(x)\\
                    f(x) + \abs{g(x) - f(x)} &g(x) < f(x)
                \end{cases}\\
                &= \frac{1}{2} \begin{cases}
                    f(x) + g(x) + \abs{f(x) - g(x)} &f(x) \geq g(x)\\
                    f(x) + g(x) + \abs{g(x) - f(x)} &g(x) < f(x)
                \end{cases}\\
                &= \cfrac{f(x) + g(x) + \abs{f(x) - g(x)}}{2}
        \end{align*}

        Since $f$ and $g$ are integrable on $[a,b]$, 
        the following is true for some $\epsilon > 0$ and partition 
        $P = \{a=x_0<x_1<\cdots<x_n=b\}$ of $[a,b]$.% with mesh less than $\cfrac{\epsilon}{4}$:
        \begin{enumerate}
            \item $U(f) = L(f)$
            \item $U(g) = L(g)$
            \item $U(f,P) - L(f,P) < \epsilon_{f}$% < \frac{\epsilon}{2}$
            \item $U(g,P) - L(g,P) < \epsilon_{g}$% < \frac{\epsilon}{2}$
            \item $\abs{f}$ is integrable.
            \item $\abs{g}$ is integrable.
            \item $U(\abs{f},P) - L(\abs{f},P) < \epsilon_{abs(f)} < \frac{\epsilon}{2}$
            \item $U(\abs{g},P) - L(\abs{g},P) < \epsilon_{abs(g)} < \frac{\epsilon}{2}$
        \end{enumerate}

        Also, by the triangular inequality,\[
            \abs{f(x) - g(x)} \leq \abs{f(x)} + \abs{g(x)}
        \] and therefore \[
            U(\abs{f(x) - g(x)}) - L(\abs{f(x) - g(x)}) < \epsilon_{abs(f-g)} < \epsilon
        \]


        %and \[
        %     \abs{f(x) - g(x)} \geq f(x) - g(x)
        % \]

        \begin{align*}
            U(h,P) - L(h,P) 
                &= \sum_{i=1}^n M(h,[x_{x-1},x_{i}])(x_{i}-x_{i-1}) 
                    - \sum_{i=1}^n m(h,[x_{x-1},x_{i}])(x_{i}-x_{i-1})\\
                &= \sum_{i=1}^n h(x_i) (x_{i}-x_{i-1}) 
                    - \sum_{i=1}^n h(x_{i-1}) (x_{i}-x_{i-1})\\
                &= \sum_{i=1}^n [h(x_i) - h(x_{i-1})] (x_{i}-x_{i-1})\\
                &= \sum_{i=1}^n \qty[
                    \cfrac{
                        \qty(f(x_{i}) + g(x_{i}) + \abs{f(x_{i}) - g(x_{i})})
                        - \qty(f(x_{i-1}) + g(x_{i-1}) + \abs{f(x_{i-1}) - g(x_{i-1})})
                    }{2}] (x_{i}-x_{i-1})\\
                &= \frac{1}{2} \sum_{i=1}^n \qty[
                    \qty(f(x_{i}) - f(x_{i-1})) ] (x_{i}-x_{i-1})\\
                    &\qquad+ \qty[g(x_{i}) - g(x_{i-1})] (x_{i}-x_{i-1})\\
                    &\qquad+ \qty[\abs{f(x_{i}) - g(x_{i})} - \abs{f(x_{i-1}) - g(x_{i-1})}
                    ] (x_{i}-x_{i-1})\\ 
                &= \frac{1}{2} \sum_{i=1}^n \qty[f(x_{i}) - f(x_{i-1})] (x_{i}-x_{i-1})\\
                    &\qquad + \frac{1}{2} \sum_{i=1}^n \qty[g(x_{i}) - g(x_{i-1})] (x_{i}-x_{i-1})\\
                    &\qquad + \frac{1}{2} \sum_{i=1}^n \qty[
                        \abs{f(x_{i}) - g(x_{i})} - \abs{f(x_{i-1}) - g(x_{i-1})}
                    ] (x_{i}-x_{i-1})\\
                &= \frac{1}{2} \qty(U(f,P) - L(f,P)) 
                        + \frac{1}{2} \qty(U(g,P) - L(g,P))
                        + \frac{1}{2} \qty(U(\abs{f-g},P) - L(\abs{f-g},P))\\
            U(h,P)-L(h,P) 
                &< \frac{1}{2} \qty(\frac{\epsilon}{2})
                    + \frac{1}{2} \qty(\frac{\epsilon}{2})
                    + \frac{1}{2} \qty(\epsilon)
                = \epsilon
        \end{align*}
        Since $\epsilon$ bounds $U(h,P) - L(h,P)$,\[
            U(h) = L(h)
        \] and therefore $h$ is integrable.
        This implies that the maximum of two integrable functions, 
        $\max{f,g}$, 
        will always be integrable.
    \end{proof}
\end{theorem}




% Problem 5 ----------------------------------------------
\newpage
\section{}
Suppose $f$ and $g$ are continuous functions on $[a,b]$ such that $\int_{a}^{b} f = \int_{a}^{b} g$. 
Prove there exists $x$ in $(a,b)$ such that $f(x) = g(x)$.

\begin{theorem}
    If functions $f:[a,b]\to\R$ and $g:[a,b]\to\R$ are continuous on $[a,b]$,
    and $\int_{a}^{b} f = \int_{a}^{b} g$, then $\exists_{x\in(a,b)} \st f(x) = g(x)$.
    \begin{proof}
        Let $h:[a,b]\to\R$ be defined by \[
            h(x) = g(x) - f(x)
        \] which is known to be continuous and bounded since both $f$ and $g$ are.
        
        The equivalency of the integrals becomes \[
            \int_{a}^{b} f = \int_{a}^{b} g \implies \int_{a}^{b} g - f = \int_{a}^{b} h = 0 
        \]

        Proof by contradiction:
        First, assume $g(x) > f(x) \forall_{x\in[a,b]}$.
        This implies $h(x) > 0 \forall_{x\in[a,b]}$ and therefore $\int_{a}^{b} h > 0$ which is a contradiction.

        Similarly, assume $g(x) < f(x) \forall_{x\in[a,b]}$.
        This implies $h(x) < 0 \forall_{x\in[a,b]}$ and therefore $\int_{a}^{b} h < 0$ which is a contradiction.

        Therefore, in order for $\int_{a}^{b} h = 0$,\[
            \exists_{x_1,x_2 \in (a,b)} \st h(x_1) \leq 0 \land h(x_2) \geq 0
        \] Since $h$ is continuous, this means that \[
            \exists_{x_0\in (a,b)} \st h(x_0) = 0
        \] which implies \[
            \exists_{x_0\in (a,b)} \st g(x_0) = f(x_0)
        \]
    \end{proof}
\end{theorem}

\end{document}
