% Standard Article Definition
\documentclass[]{article}

% Page Formatting
\usepackage[margin=1in]{geometry}
\setlength\parindent{0pt}

% Graphics
\usepackage{graphicx}

% Math Packages
\usepackage{physics}
\usepackage{amsmath, amsfonts, amssymb, amsthm}
\usepackage{mathtools}

% Extra Packages
\usepackage{listings}
\usepackage{hyperref}

% Section Heading Settings
\usepackage{enumitem}
\renewcommand{\theenumi}{\alph{enumi}}
\renewcommand*{\thesection}{Problem \arabic{section}}
\renewcommand*{\thesubsection}{\alph{subsection})}
\renewcommand*{\thesubsubsection}{}%\quad \quad \roman{subsubsection})}

%Custom Commands
\newcommand{\Rel}{\mathcal{R}}
\newcommand{\R}{\mathbb{R}}
\newcommand{\C}{\mathbb{C}}
\newcommand{\N}{\mathbb{N}}
\newcommand{\Z}{\mathbb{Z}}
\newcommand{\Q}{\mathbb{Q}}

\newcommand{\toI}{\xrightarrow{\textsf{\tiny I}}}
\newcommand{\toS}{\xrightarrow{\textsf{\tiny S}}}
\newcommand{\toB}{\xrightarrow{\textsf{\tiny B}}}

\newcommand{\divisible}{ \ \vdots \ }
\newcommand{\st}{\ : \ }

% Theorem Definition
\newtheorem{definition}{Definition}
\newtheorem{assumption}{Assumption}
\newtheorem{theorem}{Theorem}
\newtheorem{lemma}{Lemma}
\newtheorem{proposition}{Proposition}
\newtheorem{example}{Example}


%opening
\title{MATH 5302 Elementary Analysis II - Homework 7}
\author{Jonas Wagner}
\date{2022, April 13\textsuperscript{th}}

\begin{document}

\maketitle

\section*{Preliminaries}
% \begin{definition}
%     \textbf{\emph{Darboux-Stieltjes Integral}}
%     Let $f : [a,b] \to \R$ and $\alpha : [a,b] \to \R$, with $f$ bounded and $\alpha$ increasing on $[a,b]$.
%     Let partition $P$ be defined as \[
%         P = \qty{a = x_0 < x_1 < \cdots < x_{n-1} < x_n = b}
%     \] Let \[
%         M(f, [x_1, x_2]) = \sup_{x \in [x_1, x_2]} f(x)
%     \] and \[
%         m(f, [x_1, x_2]) = \inf_{x \in [x_1, x_2]} f(x)
%     \]
%     \begin{enumerate}
%         \item The upper and lower \emph{\underline{Darboux-Stieltjes Sums}} are defined \[            
%             U(f, \alpha, P) = \sum_{i=1}^{n} M(f, [x_{i-1}, x_i]) \cdot \Delta_i \alpha
%         \] and \[
%             L(f, \alpha, P) = \sum_{i=1}^{n} m(f, [x_{i-1}, x_i]) \cdot \Delta_i \alpha
%         \] respectively with \[
%             \Delta_i \alpha = \alpha(x_i) - \alpha(x_{i-1})
%         \] A more general sum $S(f, \alpha, P)$ is when $f(x_i^*)$ for $x_i^* \in [x_{i-1}, x_{i}]$ is used instead.

%         \emph{\textbf{\underline{Note:}}} \[
%             m(f,[a,b]) \cdot (\alpha(b) - \alpha(a)) 
%             \leq L(f, \alpha, P) 
%             \leq U(f, \alpha, P)
%             \leq M(f,[a,b]) \cdot (\alpha(b) - \alpha(a)) 
%         \]
%         \item The upper and lower \emph{\underline{Darboux-Stieltjes Integrals}} are defined \[            
%             U(f, \alpha) = \inf_{P \text{ partition of } [a,b]} U(f, \alpha, P)
%         \] and \[
%             L(f, \alpha) = \sup_{P \text{ partition of } [a,b]} U(f, \alpha, P)
%         \] respectively.

%         \emph{\textbf{\underline{Note:}}}\[
%             L(f, \alpha)
%             \leq L(f, \alpha, P) 
%             \leq U(f, \alpha, P)
%             \leq U(f, \alpha)
%         \] for any $P$ partition of $[a,b]$.
%         \item $f$ is called \emph{\underline{Darboux-Stieltjes Integrable}} with respect to $\alpha$ if and only if \[
%             \forall_{\epsilon>0} \exists_{P = \qty{a = x_0 < x_1 < \cdots < x_{n-1} < x_n = b}} : 
%             U(f,\alpha,P) - L(f,\alpha,P) < \epsilon
%         \] in which case the \emph{\underline{Darboux-Stieltjes Integral}} with respect to $\alpha$ is defined as \[
%             \mathcal{DS} \int_{a}^{b} f \dd{\alpha} = U(f,\alpha) = L(f, \alpha)
%         \]
%         \emph{\textbf{\underline{Note:}}}
%         If $f$ is also continuous on $[a,b]$ then $f$ is Riemann-Stieljes integrable which implies $f$ is Darboux-Stieljes integrable.
%     \end{enumerate}
%     \textbf{\emph{\underline{Properties:}}}
%     When $f$ is Darboux-Stieltjes integrable on $[a,b]$ and $\alpha$ is increasing on $[a,b]$ then
%     \begin{enumerate}
%         \item $\abs{f}$ is Darboux-Stieltjes integrable on $[a,b]$ and \[
%             \mathcal{DS} \int_{a}^{b} f \dd{\alpha} \leq \mathcal{DS} \int_{a}^{b} \abs{f} \dd{\alpha}
%         \]
%         \item $f^2$ is Darboux-Stieltjes integrable on $[a,b]$.
%         \item If $g$ is also Darboux-Stieltjes integrable on $[a,b]$, then $fg$ is Darboux-Stieltjes integrable on $[a,b]$.
%         \item For $\alpha_1$ and $\alpha_2$ also increasing on $[a,b]$ and $f$ is Darboux-Stieltjes integrable with respect to $\alpha_1$ and $\alpha_2$, then $f$ is Darboux-Stieltjes integrable with respect to $\alpha_1$ and $\alpha_2$. 
%         Additionally, \begin{align*}
%             & \mathcal{DS} \int_{a}^{b} f(x) \dd{\alpha_1(x)} + \mathcal{DS} \int_{a}^{b} f(x) \dd{\alpha_2(x)}\\
%             = & \mathcal{DS} \int_{a}^{b} f(x) \dd{\alpha(x) + \alpha_2(x)}
%         \end{align*}
%         \item For $a < c < b$, $f$ is Darboux-Stieltjes integrable with respect to $\alpha$ on $[a,b]$ if and only if $f$ is Darboux-Stieltjes integrable with respect to $\alpha$ on $[a,c]$ and $[c,b]$. 
%         Furthermore, \[
%             \mathcal{DS} \int_{a}^{b} f(x) \dd{\alpha(x)}
%             = \mathcal{DS} \int_{a}^{c} f(x) \dd{\alpha(x)} 
%             + \mathcal{DS} \int_{c}^{b} f(x) \dd{\alpha(x)}
%         \]
%     \end{enumerate}
% \end{definition}

% \begin{definition}
%     \textbf{Continuity:}
%     Let $f : [a,b] \to \R$. 
%     \begin{enumerate}
%         \item $f$ is \emph{\underline{Lipschitz Continuous}} on $[a,b]$ if \[
%             \exists_{C} : \forall_{x,y \in [a,b]} \abs{f(x) - f(y)} \leq \abs{x - y} 
%         \]
%         \item $f$ is \emph{\underline{Absolutely Continuous}} on $[a,b]$ if \[
%             \forall_{\epsilon>0} \exists_{\delta>0}
%             \forall_{
%                 \text{finite collection } \qty{(x,x')} \text{ of nonoverlaping intervals} 
%                 : \sum_{i=1}^{n} \abs{x_i' - x_i} < \delta
%             } \sum_{i=1}^{n} \abs{f(x_i') - f(x_i)} < \epsilon
%         \]
%         \item $f$ is \underline{\emph{uniformly continuous}} on $[a,b]$ if \[
%             \forall_{\epsilon>0} \exists_{\delta>0} : 
%                 \qty(x,y \in [a,b]) \land \qty{\abs{x-y} < \delta}
%             \implies \abs{f(x) - f(y)} < \epsilon
%         \]
%         \item $f$ is \emph{\underline{continuous}} on $[a,b]$ if $f$ is continuous at all $x_0 \in [a,b]$.
%         i.e. \[
%             \forall_{\epsilon > 0} \exists_{\delta>0} : 
%             \forall_{x \in [a,b]} \land \abs{x - x_0} < \delta
%             \implies \abs{f(x) - f(x_0)} < \epsilon
%         \]
%     \end{enumerate}
%     \emph{\underline{\textbf{Properties:}}}
%     \begin{enumerate}
%         \item $f$ continuous on closed $[a,b]$, then $f$ is uniformly continuous on $[a,b]$.
%         \item $f$ differentiable at $x \in [a,b]$ implies Locally Lipschitz continuous at $x$.
%         \item $C^1[a,b]$ is the set of differentiable functions with continuous derivatives on $[a,b]$.
%         \item $C^1[a,b]$ $\subset$ differentiable functions with bounded derivatives
%         \item Differentiable with bounded derivatives 
%         $\implies$ Lipschitz continuous 
%         $\implies$ Absolutely continuous
%         $\implies$ uniformly continuous
%         $\implies$ continuous
%     \end{enumerate}
% \end{definition}


\begin{definition}
    \underline{\emph{$n$-dimensional Euclidean norm-space:}}
    Let $\R^n$ be defined as \[
        \R^n := \{
            x = (x_1, x_2, \dots, x_n) 
            \st x_j \in \R
        \}
    \] Let $A, B \subseteq \R^n$.
    \begin{enumerate}
        \item \emph{\underline{Compliment of $A$}} \[
            A^c := \{
                x \in \R^n \st x \neq A
            \}
        \] \item \emph{\underline{Union of $A$ and $B$}} \[
            A \cup B := \{
                x \in \R^n \st x \in A \lor x \in B
            \}
        \] \item \emph{\underline{Intersection of $A$ and $B$}} \[
            A \cap B := \{
                x \in \R^n \st x \in A \land x \in B
            \}
        \] \item \emph{\underline{Difference of $A$ and $B$}}\[
            A \backslash B = A \cap B^c := \{
                x \in \R^n \st x \in A \land x \neq B
            \}
        \] \item \emph{\underline{Closure of $A$}}\[
            \overline{A} := \{
                x \in \R^n \st x \in A \lor x = \lim_{k\to\infty} x_k \st [x_k] \in A
            \}
        \] \item \emph{\underline{Euclidean norm on $\R^n$}} \[
            \norm{x} := \sqrt{x_1^2 + x_2^2 + \cdots + x_n^2}
        \] with triangular inequality: \[
            \norm{x+y} \leq \norm{x} + \norm{y}
        \] \item \emph{\underline{Metric on $\R^n$}} \[
            d(x,y) = \norm{x - y}
        \] with properties \begin{enumerate}
            \item $d(x,y) \geq 0$
            \item $d(x,y) = 0 \iff x = y$
            \item $d(x,y) = d(y,x)$
            \item $d(x,y) \leq d(x,z) + d(z,y)$
        \end{enumerate}
        \item $A$ is considered \emph{\underline{bounded}} if every point in $A$ is bounded: \[
            \exists_{M > 0} \st \forall_{x\in A} \norm{x} \leq M
        \]
    \end{enumerate}
\end{definition}

\begin{definition}
    \emph{\textbf{Open and Closed Sets:}} 
    Let $A \subseteq \R^n$ and $x \in A$.
    \begin{enumerate}
        \item Denote the \emph{\underline{open ball}} centered at $x$ of radius $r$ as: \[
            B(x,r) := \{
                y \in \R^n \st d(x,y) < r
            \}
        \] \item $x$ is considered an \emph{\underline{interior point of $A$}} if :\[
            \exists_{r > 0} \st B(x,r) \subseteq A
        \] \item  $A$ is considered \emph{\underline{open}} if every point $x \in A$ is an interior point of $A$: \[
            \forall_{x \in A} \exists_{r>0} \st B(x,r) \subseteq A
        \] The following properties exist for open sets:\begin{enumerate}
            \item $\emptyset$ is open
            \item $\R^n$ is open
            \item Union of any collection of open sets is open
            \item Intersection of any finite collection of open sets is open
            \item Any open ball is an open set
        \end{enumerate}
        \item The \emph{\underline{Interior of $A$}} is the set of all interior points of $A$ \[
            A^\circ := \{
                x \st x \text{is an interior point of $A$}
            \}
        \] Properties of $A^\circ$ \begin{enumerate}
            \item $A$ open $\iff$ $A^\circ = A$
            \item $A^\circ$ open
            \item $(A^\circ)^\circ = A^\circ$
            \item $(A \cap B)^\circ = A^\circ \cap B^\circ$
            \item $A^\circ \cup B^\circ$ not generally equal to $(A \cup B)^\circ$
            \item $A^\circ$ is the union of all open subsets of $A$
            \item $A^\circ$ is the largest open subset of $A$
        \end{enumerate}
        \item The \emph{\underline{Boundary of $A$}} is defined as the closure minus the Interior of $A$:\[
            \partial A := \overline{A} \backslash A^\circ
        \] \item $A$ is considered \emph{\underline{closed}} if $A^c$ is open. 
        Properties of closed sets \begin{enumerate}
            \item $\R^n$ is closed
            \item $\emptyset$ is closed
            \item The intersection of any collection of closed sets is closed
            \item The union of any finite collection of closed sets is closed
        \end{enumerate}
    \end{enumerate}
\end{definition}

\begin{definition}
    \emph{\textbf{Compact set:}} 
    Let $A \subseteq \R^n$. 
    \begin{enumerate}
        \item 
        \item $A$ is called \emph{\underline{compact}} if every open cover of $A$ has a finite subcover.
        \item Properties of compact sets \begin{enumerate}
            \item $\emptyset$ is compact
            \item Any finite set is compact
            \item $A$ and $B$ compact $\implies$ $A \cup B$ compact
            \item Any finite union of compact sets is compact
            \item $B(x,r)$ is not compact
            \item $\R^n$ is not compact
            \item If $A$ is compact then $A$ is closed and bounded.
        \end{enumerate}
    \end{enumerate}
\end{definition}

\begin{definition}
    \emph{\textbf{Lebesgue Measure:}}
    Let $A \subseteq \R^n$. 
    \emph{Note:} 
    For various $n$, a Lebesgue Measure is essentially:\begin{enumerate}
        \item For $n=1$, $\lambda(A)$ is a length
        \item For $n=2$, $\lambda(A)$ is an area
        \item For $n=3$, $\lambda(A)$ is a volume
    \end{enumerate}
    Each of the following stages define the Lebesgue measure in increasing complexity. \begin{enumerate}
        \item \emph{\textbf{Empty Set:}} \[
            \lambda(\emptyset) = 0
        \] 
        \item \emph{\textbf{Special Rectangles:}}
        
        Let \[
            I = [a_1, b_1] \cross \cdots \cross [a_n, b_n]
        \] then \[
            \lambda(I) = (b_1 - a_1) \cdots (b_n - a_n)
        \]
        \item \emph{\textbf{Special Polygons:}}
        Special polygons are a finite union of special rectangles. 
        They are closed and bounded subsets and therefore compact. 
        
        Let $P$ be a special polygon, decomposed into the following union of nonoverlaping special rectangles: \[
            P = \bigcup_{k=1}^{N} I_k
        \] The Lebesgue Measure for the special polygon is defined as the sum of the Lebesgue Measures of the special rectangles:\[
            \lambda(P) = \sum_{k=1}^{N} \lambda(I_k)
        \] The following are a few properties of the Lebesgue Measure for Special Polygons:\begin{enumerate}
            \item $P_1 \subseteq P_2 \implies \lambda(P_1) \leq \lambda(P_2)$
            \item $P_1 \cap P_2 = \emptyset \implies \lambda(P_1 \cup P_2) = \lambda(P_1) + \lambda(P_2)$
        \end{enumerate}
        \item \emph{\textbf{Open Sets:}}
        Let $G \subseteq \R^n$ be open and $G \neq \emptyset$. 

        The Lebesgue Measure of an open set is defined as \[
            \lambda(G) = \sup \{
                \lambda(P) \st P \subset G, \ P \textnormal{ is a special polygon}
            \}
        \] The following are some properties of the Lebesgue Measure for open sets: \begin{enumerate}
            \item $0 \leq \lambda(G) \leq \infty$
            \item $\lambda(G) = 0 \iff G = \emptyset$
            \item $\lambda(\R^n) = \infty$
            \item $G_1 \subseteq G_2 \implies \lambda(G_1) \leq \lambda(G_2)$
            \item $\lambda\qty(\bigcup_{k=1}^{\infty} G_k) \leq \sum_{k=1}^{\infty} \lambda(G_k)$
            \item $\bigcap_{k=1}^{\infty} G_k = \emptyset \implies \lambda\qty(\bigcup_{k=1}^{\infty} G_k) = \sum_{k=1}^{\infty} \lambda(G_k)$
            \item If $P$ is a special polygon, then $\lambda(P) = \lambda(P^\circ)$
        \end{enumerate}
        \item \emph{\textbf{Compact Sets:}}
        Let $K \subseteq \R^n$ be a compact set.

        The Lebesgue Measure of a compact set is defined as \[
            \lambda(K) = \inf \{
                \lambda(G) \st K \subset G, \ G \textnormal{ is an open set}
            \}
        \] The following are some properties of the Lebesgue Measure for compact sets: \begin{enumerate}
            \item $0 \leq \lambda(K) \leq \infty$
            \item $K_1 \subseteq K_2 \implies \lambda(K_1) \leq \lambda(K_2)$
            \item $\lambda(K_1 \cup K_2) \leq \lambda(K_1) + \lambda(K_2)$
            \item $K_1 \cap K_2 = \emptyset \implies \lambda(K_1 \cup K_2) = \lambda(K_1) + \lambda(K_2)$
        \end{enumerate}
    \end{enumerate}
\end{definition}







% Problem 1 ----------------------------------------------
\newpage
\section{}
\subsection*{Problem:}
Let $A \subset \R^n$ be a compact set. 
Show that $A$ is bounded. 

\subsection*{Solution:}
\begin{theorem}
    Let $A \subset \R^n$. 
    If $A$ is a compact set then $A$ is also bounded.
    \begin{proof}
        For $x \in A$, let \[
            G_{i} = B(x,r_i) = \{
                y \in \R^n \st d(x,y) < r_{i}
            \}, \forall_{i \in \N}
        \] with $r_i \leq r_{i+1}$, meaning $G_i \subset G_{i+1}$ and also that \[
            A \subseteq \bigcup_{i=1}^{\infty} G_i
        \] $A$ compact means that the finite subcover exists \[
            \exists_{i_1,i_2,\dots,i_N} \st A \subseteq \bigcup_{i=1}^{N} G_{i_k}
        \] and therefore \[
            \exists_k \st A \subseteq G_{i_k}
        \] which means that \[
            \exists_{r_k} \st \forall_{y \in A} d(x,y) \leq r_k
        \] which is the definition of a bounded set.
    \end{proof}
\end{theorem}

% Problem 2 ----------------------------------------------
\newpage
\section{}
\subsection*{Problem:}
Let $G$ be a nonempty subset of $\R^n$. 
If $G$ is open and $P$ is a special polygon with $P\subset G$, prove there exists a special polygon $P'$ such that $P \subset P' \subset G$ and $\lambda(P) < \lambda(P')$. 
(Hint: consider $G \backslash P$).

\subsection*{Solution:}
\begin{theorem}
    Let $G \neq \emptyset \subset \R^n$ and $P \subset G$.
    If $G$ open and $P$ is a special polygon, then \[
        \exists_{P'} \st P \subset P' \subset G \land \lambda(P) < \lambda(P')
    \]
    \begin{proof}
        By definition, $P$ is composed of a finite collection of nonoverlaping special rectangles:\[
            P = \bigcup_{k=1}^{N} I_k
        \] Since $G$ is open and $P \subset G$, \[
            G \backslash P \neq \emptyset \implies \exists_{I_{N+1} \textnormal{ special rectangle}} \subset G \backslash P
        \] $I_{N+1}$ is then another nonoverlaping special rectangle so that \[
            P' = P \cup I_{N+1} = \bigcup_{k=1}^{N+1} I_k \subset G
        \] Since $\lambda(P) = \sum_{k=1}^{N} \lambda(I_k)$, \[
            \lambda(P') = \lambda(P) + \lambda(I_k) > P
        \] therefore \[
            P \subset P' \subset G \land \lambda(P) < \lambda(P')
        \]
    \end{proof}
\end{theorem}

% Problem 3 ----------------------------------------------
\newpage
\section{}
\subsection*{Problem:}
Use the definition of Lebesgue measure, $\lambda(G)$, of an open set $G \subset \R^n$ to prove the following statements: \begin{enumerate}
    \item If $G$ is a bounded open set, then $\lambda(G) < \infty$.
    \item Let \[
        G = \{
            (x,y) \in \R^2 \st 0 < x < 1 \land 0 < y < x^2
        \}
    \] Then $\lambda(G) = \frac{1}{3}$.
\end{enumerate}
(Hint: 
relate $\lambda(G)$ to the lower and upper Darboux sums of the function $f(x) = x^2$ on $[0,1]$. 
However, you cannot use the methods of calculus to the extent that $\lambda(G) = \int_0^1 x^2 \dd{x} = \frac{1}{3}$.
You must use the actual definition of $\lambda(G)$).

\subsection{If $G$ is a bounded open set, then $\lambda(G) < \infty$.}
\begin{theorem}
    Let $G \subset \R^n$.
    If $G$ is a bounded open set then $\lambda(G) < \infty$.
    \begin{proof}
        The definition of $\lambda(G)$ as an open set is \[
            \lambda(G) = \sup \{
                \lambda(P) \st P \subset G, \ P \textnormal{ is a special polygon}
            \}
        \] Let $P_k$ be defined as a sequence with $P_k = \cup_{i=1}^{k} I_i \subset P_{k+1} \subset G \forall_{k \in \N}$.
        Since $G$ is bounded, every $P_{k} \subset G$ would also be bounded (which is enough to conclude that $\lambda(P_k) \forall_{k}$ is bounded).
        We have $\lambda(P_k) \leq \lambda(P_{k+1})$, and more specifically,\[
            \lambda(P_{k+1}) = \lambda(P_{k}) + \lambda(I_{k+1})
        \] Since $I_{k+1} \cup G \backslash P_{k+1} = G \backslash P_{k} \subset G \backslash P_{k-1}$, \[
            \lambda(G \backslash P_{k}) = \lambda(G \backslash P_{k+1}) + \lambda(I_{k+1})
        \] and therefore $\lambda(I_{k+1}) < \lambda(I_{k})$. 
        This means that a finite upper bound exists on $P_k$ as \[
            \sup{P_k} = \lim_{k\to \infty} \sum_{i=1}^k I_i
        \] and thus $\lambda(G) = \sup{P_k} < \infty$.
    \end{proof}
\end{theorem}

\subsection{
    Let $G = \{(x,y) \in \R^2 \st 0 < x < 1 \land 0 < y < x^2\}$
    Then $\lambda(G) = \frac{1}{3}$.
}
\begin{example}
    Let \[
        G = \{(x,y) \in \R^2 \st 0 < x < 1 \land 0 < y < x^2\}
    \]
    $\lambda(G) = \frac{1}{3}$.
    \begin{proof}
        Consider $f : [0,1] \to \R$ defined as \[
            f(x) = x^2
        \] Defining a partition of $[0,1]$ \[
            \mathcal{P}_k = \{
                0 = x_0 < x_1 < \cdots < x_k = 1
            \}
        \] Since $f$ is increasing $\forall_{x\in[0,1]}$, special polygons $P_{l_k} \subset G \subset P_{u_k}$ can be defined as the union of special rectangles:\[
            P_{l_k} = \bigcup_{i=1}^{k} I_{l_i}, \ I_i = [x_{i-1}, x_i] \cross [0,f(x_{i-1})]
        \] and \[
            P_{u_k} = \bigcup_{i=1}^{k} I_{u_i}, \ I_i = [x_{i-1}, x_{i}] \cross [0,f(x_{i})]
        \] We then have the Lebesgue Measures for the upper and lower special polygons as\[
            \lambda(P_{l_k}) 
            = \sum_{i=1}^{k} \lambda(I_{l_k}) 
            = \sum_{i=1}^{k} (f(x_{i-1}) - 0) \cdot (x_{i} - x_{i-1})
            = L(f, P_k)
        \] and \[
            \lambda(P_{u_k}) 
            = \sum_{i=1}^{k} \lambda(I_{u_k}) 
            = \sum_{i=1}^{k} (f(x_{i}) - 0) \cdot (x_{i} - x_{i-1})
            = U(f, P_k)
        \] where each are equivalent to their respective Darboux sums. 
        Therefore, \[
            \lambda(P_{l_k}) = L(f, P_k) \leq \sup_{P_k} \lambda(P_k) = L(f) \leq U(f) = \inf_{P_k} \lambda(P_k) \leq U(f,P_k) = \lambda(P_{u_k})
        \]
        Since we know that these Darboux sums result in equivalent Darboux integrals, $L(f) = U(f) = \mathcal{R}\int_{0}^{1} f = \frac{1}{3}$, we can say that \[
            \lambda(P_{l_\infty}) = \lambda(G) = \lambda(P_{u_\infty}) = \frac{1}{3}
        \]
    \end{proof}
\end{example}

% Problem 4 ----------------------------------------------
\newpage
\section{}
\subsection*{Problem:}
Prove that every nonempty open subset of $\R$ can be expressed as a countable disjoint union of open intervals:\[
    G = \bigcup_{k} (a_k,b_k)
\] where the range on $k$ can be finite or infinite.
Furthermore, show that this expression is unique except for the number of the component intervals.
(Hint:
for any $x\in G$, show that there exist a largest open interval $A_x$ such that $x\in A_x$ and $A_x \subseteq G$.
Also note that the set of rational numbers is countable and dense in $\R$.
)

\subsection*{Solution:}
\begin{theorem} \label{thm:pblm4}
    Let $G \in \R$ be nonempty and open. 
    $G$ can be expressed as a countable disjoint union of open intervals: \[
        G = \bigcup_{k} (a_k, b_k)
    \]
    \begin{proof}
        Let $x \in G$. 
        The largest open interval within $G$ containing $x$ is defined as \[
            A_x := (a_x,b_x) \subseteq G 
                \st x \in (a_x,b_x)
                \land \qty(
                    \forall_{(a_x',b_x') \subseteq G} x \in (a_x',b_x') \implies (a_x',b_x') \subset (a_x,b_x)
                )
        \] From this definition we have the following for $A_x = (a_x, b_x)$, \[
            A_x = (a_x, b_x) \subseteq G \implies a_x \in \overline{G} \land b_x \in \overline{G}
        \] Furthermore, the definition that $A_x$ is the maximum possible subset implies that \[
            a_x \in \partial G \land b_x \in \partial G
        \] therefore \[
            a_x < x < b_x \st a_x, b_x \in \partial G \land 
        \] Since $G \subseteq \R$, these boundary points will be unique. 
        i.e. \[
            \forall_{x \in G} \exists_{\textnormal{unique }a_x, b_x \in \partial G} \st a_x < x < b_x
        \] 

        Consider $K = \{x_k \in \Q\}$. 
        Since $\Q$ is complete in $\R$, and therefore complete in $G$, we have that \[
            \exists_{K \subset \Q} \bigcup_{x_k \in K} A_{x_k} = G
        \]
        This is because every open region $A_k = (a_k,b_k) \in G$ will contain a point $x_k \in K$ which will result in the union of all these countable open regions to be equivalent to $G$.
    \end{proof}
\end{theorem}

% Problem 5 ----------------------------------------------
\newpage
\section{}
\subsection*{Problem:}
In the notation of Problem 4, prove that $\lambda(G) = \sum_{k} (b_k - a_k)$.

\subsection*{Solution:}
\begin{theorem}
    Let $G \subseteq \R$ be nonempty and open. 
    The Lebesgue Measure of $G$ is defined as \[
        \lambda(G) = \sum_{k} (b_k - a_k)
    \]
    \begin{proof}
        From Theorem \ref{thm:pblm4} we have that \[
            G = \bigcup_{k} (a_k, b_k)
        \] Since $(a_k, b_k) \subset \R$, we know that \[
            \lambda((a_k, b_k)) = b_k - a_k
        \] Since $G$ is composed of a collection of disjoint open sets, we have that \[
            \lambda(\bigcup_{k} (a_k, b_k)) = \sum_{k} \lambda((a_k, b_k))
        \] Therefore, \[
            \lambda(G) = \sum_{k} (b_k - a_k)
        \]
    \end{proof}
\end{theorem}

% Problem 6 ----------------------------------------------
\newpage
\section{}
\subsection*{Problem:}
Let $C$ be the Cantor Set. 
Show that $\frac{1}{4} \in C$ and that $\frac{1}{4}$ is not an end point of any of the intervals in the $G_k$'s.

\subsection*{Preliminaries:}
\begin{definition}
    The \underline{\emph{Cantor set}} $C$ is defined by \[
        C = [0,1] \backslash \bigcup_{k=1}^{\infty} G_k
    \] where $G_k$ are defined as \[
        \bigcup_{i=0}^{3^{k-1} - 1} \qty(\frac{3i + 1}{3^{i+1}}, \frac{3i + 2}{3^{n+1}})
    \]
\end{definition}

\subsection*{Solution:}
\begin{example}
    Let $C$ be the Cantor Set. 
    Show that $\frac{1}{4} \in C$ and $\frac{1}{4} \notin \partial G_k \forall_{G_k}$.
    \begin{proof}
        For $n=1$, we have $C_n = C_1 = [0,1] \backslash (\frac{1}{3}, \frac{2}{3})$
        $\frac{1}{4}$ is clearly in $C_1$ since $\frac{1}{4} \in [0,1] \land \frac{1}{4} \notin (\frac{1}{3}, \frac{2}{3})$.
        
        Generally for $n$, $\frac{1}{4} \notin G_n$ since $\forall_{i \in \N} \frac{1}{4} \notin \qty(\frac{3i+1}{3^{i+1}}, \frac{3i + 2}{3^{i+1}})$. 
        Therefore, if $\frac{1}{4} \in C_n$ then $\frac{1}{4} \in C_{n+1}$ since \[
            \frac{1}{4} \in C_n 
            \implies \qty(\frac{1}{4} \in [0,1] \backslash \cup_{k=1}^{n} G_k) 
            \land \frac{1}{4} \notin G_{n+1}
        \]

        Therefore, by induction, $\frac{1}{4} \in C$.

        Additionally, since $\forall_{i \in \N} \lnot 3^{i} \divisible 4$, $\frac{1}{4} \neq \frac{3i + 1}{3^{i+1}} \neq \frac{3i+2}{3^{i+1}}$.
        Therefore, $\frac{1}{4}$ is not an endpoint of an of the intervals of $G_k$.
    \end{proof}
\end{example}

\end{document}
